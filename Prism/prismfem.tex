% !Mode:: "TeX:UTF-8"
\documentclass{article}

%%%%%%%%------------------------------------------------------------------------
%%%% 日常所用宏包

%% 控制页边距
% 如果是beamer文档类, 则不用geometry
\makeatletter
\@ifclassloaded{beamer}{}{\usepackage[top=2.5cm, bottom=2.5cm, left=2.5cm, right=2.5cm]{geometry}}
\makeatother

%% 控制项目列表
\usepackage{enumerate}
\usepackage{framed}

%% 多栏显示
\usepackage{multicol}

%% 算法环境
\usepackage{algorithm}  
\usepackage{algorithmic} 
\usepackage{float} 

%% 网址引用
\usepackage{url}

%% 控制矩阵行距
\renewcommand\arraystretch{1.4}

%% 粗体
\usepackage{bm}


%% hyperref宏包,生成可定位点击的超链接,并且会生成pdf书签
\makeatletter
\@ifclassloaded{beamer}{
\usepackage{hyperref}
\usepackage{ragged2e} % 对齐
}{
\usepackage[%
    pdfstartview=FitH,%
    CJKbookmarks=true,%
    bookmarks=true,%
    bookmarksnumbered=true,%
    bookmarksopen=true,%
    colorlinks=true,%
    citecolor=blue,%
    linkcolor=blue,%
    anchorcolor=green,%
    urlcolor=blue%
]{hyperref}
}
\makeatother



\makeatletter % 如果是 beamer 不需要下面两个包
\@ifclassloaded{beamer}{
\mode<presentation>
{
} 
}{
%% 控制标题
\usepackage{titlesec}
%% 控制目录
\usepackage{titletoc}
}
\makeatother

%% 控制表格样式
\usepackage{booktabs}

%% 控制字体大小
\usepackage{type1cm}

%% 首行缩进,用\noindent取消某段缩进
\usepackage{indentfirst}

%% 支持彩色文本、底色、文本框等
\usepackage{color,xcolor}

%% AMS LaTeX宏包: http://zzg34b.w3.c361.com/package/maths.htm#amssymb
\usepackage{amsmath,amssymb}
%% 多个图形并排
\usepackage{subfloat}
%%%% 基本插图方法
%% 图形宏包
\usepackage{graphicx}
\newcommand{\red}[1]{\textcolor{red}{#1}}
\newcommand{\blue}[1]{\structure{#1}}
\newcommand{\brown}[1]{\textcolor{brown}{#1}}
\newcommand{\green}[1]{\textcolor{green}{#1}}


%%%% 基本插图方法结束

%%%% pgf/tikz绘图宏包设置
\usepackage{pgf,tikz}
\usetikzlibrary{shapes,automata,snakes,backgrounds,arrows}
\usetikzlibrary{mindmap}
%% 可以直接在latex文档中使用graphviz/dot语言,
%% 也可以用dot2tex工具将dot文件转换成tex文件再include进来
%% \usepackage[shell,pgf,outputdir={docgraphs/}]{dot2texi}
%%%% pgf/tikz设置结束


\makeatletter % 如果是 beamer 不需要下面两个包
\@ifclassloaded{beamer}{

}{
%%%% fancyhdr设置页眉页脚
%% 页眉页脚宏包
\usepackage{fancyhdr}
%% 页眉页脚风格
\pagestyle{plain}
}

%% 有时会出现\headheight too small的warning
\setlength{\headheight}{15pt}

%% 清空当前页眉页脚的默认设置
%\fancyhf{}
%%%% fancyhdr设置结束


\makeatletter % 对 beamer 要重新设置
\@ifclassloaded{beamer}{

}{
%%%% 设置listings宏包用来粘贴源代码
%% 方便粘贴源代码,部分代码高亮功能
\usepackage{listings}

%% 设置listings宏包的一些全局样式
%% 参考http://hi.baidu.com/shawpinlee/blog/item/9ec431cbae28e41cbe09e6e4.html
\lstset{
showstringspaces=false,              %% 设定是否显示代码之间的空格符号
numbers=left,                        %% 在左边显示行号
numberstyle=\tiny,                   %% 设定行号字体的大小
basicstyle=\scriptsize,                    %% 设定字体大小\tiny, \small, \Large等等
keywordstyle=\color{blue!70}, commentstyle=\color{red!50!green!50!blue!50},
                                     %% 关键字高亮
frame=shadowbox,                     %% 给代码加框
rulesepcolor=\color{red!20!green!20!blue!20},
escapechar=`,                        %% 中文逃逸字符,用于中英混排
xleftmargin=2em,xrightmargin=2em, aboveskip=1em,
breaklines,                          %% 这条命令可以让LaTeX自动将长的代码行换行排版
extendedchars=false                  %% 这一条命令可以解决代码跨页时,章节标题,页眉等汉字不显示的问题
}

\usepackage{minted}
\renewcommand{\listingscaption}{Python code} \newminted{python}{
    escapeinside=||,
    mathescape=true,
    numbersep=5pt,
    linenos=true,
    autogobble,
    framesep=3mm} 
}
\makeatother
%%%% listings宏包设置结束


%%%% 附录设置
\makeatletter % 对 beamer 要重新设置
\@ifclassloaded{beamer}{

}{
\usepackage[title,titletoc,header]{appendix}
}
\makeatother
%%%% 附录设置结束





%% 设定行距
\linespread{1}

\newcommand{\bfm}{\boldsymbol m}
\newcommand{\bfc}{\boldsymbol c}
\newcommand{\bfd}{\boldsymbol d}
\newcommand{\bfg}{\boldsymbol g}
\newcommand{\bff}{\boldsymbol f}
\newcommand{\bfx}{\boldsymbol x}
\newcommand{\bfu}{\boldsymbol u}
\newcommand{\bfn}{\boldsymbol n}
\newcommand{\bfv}{\boldsymbol v}
\newcommand{\bft}{\boldsymbol t}
\newcommand{\bfq}{\boldsymbol q}
\newcommand{\bfs}{\boldsymbol s}
\newcommand{\bfy}{\boldsymbol y}
\newcommand{\bfA}{\boldsymbol A}
\newcommand{\bfB}{\boldsymbol B}
\newcommand{\bfC}{\boldsymbol C}
\newcommand{\bfD}{\boldsymbol D}
\newcommand{\bfT}{\boldsymbol T}
\newcommand{\bfP}{\boldsymbol P}
\newcommand{\bfI}{\boldsymbol I}
\newcommand{\bfF}{\boldsymbol F}
\newcommand{\bfK}{\boldsymbol K}
\newcommand{\bfM}{\boldsymbol M}
\newcommand{\bfS}{\boldsymbol S}
\newcommand{\bfW}{\boldsymbol W}
\newcommand{\bfG}{\boldsymbol G}
\newcommand{\bfH}{\boldsymbol H}
\newcommand{\bfQ}{\boldsymbol Q}
\newcommand{\bfJ}{\boldsymbol J}
\newcommand{\balpha}{\bm \alpha}
\newcommand{\blambda}{\bm \lambda}
\newcommand{\bsigma}{\bm \sigma}
\newcommand{\bepsilon}{\bm \epsilon}
\newcommand{\bvarepsilon}{\bm \varepsilon}
\newcommand{\btau}{\bm \tau}
\newcommand{\rmd}{\,\mathrm d}
\newcommand{\cT}{\mathcal T}
\newcommand{\cF}{\mathcal F}
\newcommand{\cS}{\mathcal S}
\newcommand{\cP}{\mathcal P}
\newcommand{\cM}{\mathcal M}
\newcommand{\cA}{\mathcal A}
\newcommand{\cE}{\mathcal E}
\newcommand{\cB}{\mathcal B}
\newcommand{\cQ}{\mathcal Q}
\newcommand{\cN}{\mathcal N}
\newcommand{\cV}{\mathcal V}
\newcommand{\cW}{\mathcal W}
\newcommand{\bbS}{\mathbb S}
\newcommand{\bbR}{\mathbb R}
\newcommand{\od}{\text{div}}
\newcommand{\os}{\text{span}}
\newcommand{\ot}{\text{tr}}
\newcommand{\norm}[1]{||#1||}
\newcommand{\dof}{\text{dof}}

%%%% 个性设置结束
%%%%%%%%------------------------------------------------------------------------


%%%%%%%%------------------------------------------------------------------------
%%%% bibtex设置

%% 设定参考文献显示风格
% 下面是几种常见的样式
% * plain: 按字母的顺序排列,比较次序为作者、年度和标题
% * unsrt: 样式同plain,只是按照引用的先后排序
% * alpha: 用作者名首字母+年份后两位作标号,以字母顺序排序
% * abbrv: 类似plain,将月份全拼改为缩写,更显紧凑
% * apalike: 美国心理学学会期刊样式, 引用样式 [Tailper and Zang, 2006]

%\makeatletter
%\@ifclassloaded{beamer}{
%\bibliographystyle{apalike}
%}{
%\bibliographystyle{abbrv}
%}
%\makeatother


%%%% bibtex设置结束
%%%%%%%%------------------------------------------------------------------------

\input{xecjk_preamble.tex}
\begin{document}
\title{三棱柱单元的实现细节及其应用}
\author{魏华祎}
\date{\chntoday}
\maketitle
\section{一般三棱柱体积的计算}

$$
\bfx = \beta_0(\lambda_0 \bfx_0 + \lambda_1 \bfx_1 + \lambda_2 \bfx_2) + 
\beta_1(\lambda_0 \bfx_3 + \lambda_1 \bfx_4 + \lambda_2 \bfx_5)
$$

其中
\begin{align*}
    \lambda_0 =& 1 - u - v \\
    \lambda_1 =& u \\
    \lambda_2 =& v \\
    \beta_0 = & 1 - w\\
    \beta_1 = & w
\end{align*}

上述变换的 Jacobi 矩阵为

$$
\scriptsize
\bfJ = \frac{\partial \bfx}{\partial \bfu}
= 
\begin{bmatrix}
    \beta_0(x_1 - x_0) + \beta_1(x_4 - x_3) & 
    \beta_0(x_2 - x_0) + \beta_1(x_5 - x_3) &
    \lambda_0(x_3 - x_0) + \lambda_1(x_4 - x_1) + \lambda_2(x_5 - x_2)\\
    \beta_0(y_1 - y_0) + \beta_1(y_4 - y_3) & 
    \beta_0(y_2 - y_0) + \beta_1(y_5 - y_3) &
    \lambda_0(y_3 - y_0) + \lambda_1(y_4 - y_1) + \lambda_2(y_5 - y_2)\\
    \beta_0(z_1 - z_0) + \beta_1(z_4 - z_3) & 
    \beta_0(z_2 - z_0) + \beta_1(z_5 - z_3) &
    \lambda_0(z_3 - z_0) + \lambda_1(z_4 - z_1) + \lambda_2(z_5 - z_2)\\
\end{bmatrix}
$$

\section{三棱柱单元上的基函数}

三角形单元上的 $k$ 次多项式空间共有 $n_k = \frac{(k+1)(k+2)}{2}$ 个基函数 
$\{\phi_i\}_{i=0}^{n_k-1}$. 区间单元上的 $k$ 次多项式空间共有$k+1$ 个基函数 $\{\varphi_j\}_{j=0}^{k}$.


三棱柱单元上的 $k$ 次多项式空间共有 $n_k(k+1)$ 个基函数.
$$
\begin{pmatrix}
    \varphi_0 \\ \varphi_1 \\ \vdots \\ \varphi_k
\end{pmatrix}
\begin{pmatrix}
    \phi_0 & \phi_1 & \cdots & \phi_{n_k-1}
\end{pmatrix}
$$

$$
\psi_{i, j}(\bfx) = \varphi_i\phi_j
$$

记 $\bfu = [u, v, w]^T$

$$
\nabla_{\bfx} \psi_{i, j}(\bfx) = J^{-1} 
\nabla_{\bfu} \psi_{i, j}(\bfx(\bfu)) 
= \varphi_i J^{-1}\nabla_{\bfu}\phi_j + \phi_j J^{-1} \nabla_{\bfu}\varphi_i
= \varphi_i \nabla_{\bfx}\phi_j + \phi_j\nabla_{\bfx}\varphi_i
$$

$$
\left[\frac{\partial \psi_{i,j}}{\partial x},\frac{\partial \psi_{i,j}}{\partial y},\frac{\partial \psi_{i,j}}{\partial z}\right]^T=J^{-1}\left[\frac{\partial \psi_{i,j}}{\partial u},\frac{\partial \psi_{i,j}}{\partial v},\frac{\partial \psi_{i,j}}{\partial w}\right]^T
$$

$$
=J^{-1}\left[\varphi_i \frac{\partial \phi_j}{\partial u},\varphi_i \frac{\partial \phi_j}{\partial v},\phi_i \frac{\partial \varphi_i}{\partial w}\right]^T
$$

$$
=J^{-1}\left[\varphi_i(\frac{\partial \phi_j}{\partial \lambda_0}\frac{\partial \lambda_0}{\partial u}+\frac{\partial \phi_j}{\partial \lambda_1} \frac{\partial \lambda_1}{\partial u}+\frac{\partial \phi_j}{\partial \lambda_2} \frac{\partial \lambda_2}{\partial u}),
\varphi_i(\frac{\partial \phi_j}{\partial \lambda_0}\frac{\partial \lambda_0}{\partial v}+\frac{\partial \phi_j}{\partial \lambda_1} \frac{\partial \lambda_1}{\partial v}+\frac{\partial \phi_j}{\partial \lambda_2} \frac{\partial \lambda_2}{\partial v}),
\phi_j(\frac{\partial \varphi_i}{\partial \beta_0} \frac{\partial \beta_0}{\partial w}+\frac{\partial \varphi_i}{\partial \beta_1} \frac{\partial \beta_1}{\partial w})\right]^T
$$

$$
=J^{-1}\left[\varphi_i(-\frac{\partial \phi_j}{\partial \lambda_0}+\frac{\partial \phi_j}{\partial \lambda_1}),\varphi_i(-\frac{\partial \phi_j}{\partial \lambda_0}+\frac{\partial \phi_j}{\partial \lambda_2}),\phi_j(-\frac{\partial \varphi_i}{\partial \beta_0}+\frac{\partial \varphi_i}{\partial \beta_1})\right]^T
$$

$$J^{-1}=\frac{\begin{vmatrix}
a_{11} & a_{12} & a_{13}\\
a_{21} & a_{22} & a_{23}\\
a_{31} & a_{32} & a_{33}
\end{vmatrix}
}{|J|}$$

三棱柱单元的基函数是由三角单元上基函数和区间单元上基函数张成的,当我们再给定$\varphi$时可以确定三棱柱中间的一个三角形截面。同理,当我们给定$\phi$时,我们可以确定三角形截面上一个点$\bfx$,过此点可以确定一条直线$l$。



1.现在三角形基函数$\phi$沿着三角形截面的梯度可以计算出,我们设为已知,且$\phi$沿着前面直线$l$的导数为$0$,我们考虑$\phi$在$\bfx$点处的空间梯度是多少?

如果三角形单元在$xoy$平面上或者平行$xoy$平面,那么对$z$求导会为$0$
$$\nabla_{\bfx} \phi =[\frac{\partial \phi}{\partial x},\frac{\partial \phi}{\partial y},\frac{\partial \phi}{\partial z}]=[\frac{\partial \phi}{\partial x},\frac{\partial \phi}{\partial y},0]$$


如果三角单元不在$xoy$平面,那么基函数与$z$有关($\phi$是也是用面积比求的,不过会与$z$有关)
$$\nabla_{\bfx} \phi =[\frac{\partial \phi}{\partial x},\frac{\partial \phi}{\partial y},\frac{\partial \phi}{\partial z}]$$

2.同理对于区间上基函数$\varphi$沿着直线$l$的导数可以算出,我们设为已知,且因为在三角形截面上因为$\beta_0,\beta_1$为常数,所以求导为$0$。我们考虑$\varphi$在$\bfx$处的空间梯度是多少?


如果区间单元平行$z$轴,那么$\varphi$与$x,y$无关,求导为$0$
$$\nabla_{\bfx} \varphi =[\frac{\partial \varphi}{\partial x},\frac{\partial \varphi}{\partial y},\frac{\partial \varphi}{\partial z}]=[0,0,\frac{\partial \varphi}{\partial z}]$$

如果区间单元不平行$z$轴,那么$\varphi$与$x,y$有关
$$\nabla_{\bfx} \varphi =[\frac{\partial \varphi}{\partial x},\frac{\partial \varphi}{\partial y},\frac{\partial \varphi}{\partial z}]$$



$$\varphi_i \nabla_{\bfx}\phi_j + \phi_j\nabla_{\bfx}\varphi_i=\varphi_i[\frac{\partial \phi_j}{\partial x},\frac{\partial \phi_j}{\partial y},\frac{\partial \phi_j}{\partial z}]+\phi_j[\frac{\partial \varphi_i}{\partial x},\frac{\partial \varphi_i}{\partial y},\frac{\partial \varphi_i}{\partial z}]$$

%=\varphi_i[\frac{\partial \phi_j}{\partial x},\frac{\partial \phi_j}{\partial y},0]+\phi_j[0,0,\frac{\partial \phi_j}{\partial z}]=[\varphi_i\frac{\partial \phi_j}{\partial x},\varphi_i\frac{\partial \phi_j}{\partial y},\phi_j\frac{\partial \phi_j}{\partial z}]

\cite{sheng2008}
\bibliographystyle{abbrv}
\bibliography{ref}
\end{document}
