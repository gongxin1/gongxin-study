% !Mode:: "TeX:UTF-8"
\documentclass{article}

%%%%%%%%------------------------------------------------------------------------
%%%% 日常所用宏包

%% 控制页边距
% 如果是beamer文档类, 则不用geometry
\makeatletter
\@ifclassloaded{beamer}{}{\usepackage[top=2.5cm, bottom=2.5cm, left=2.5cm, right=2.5cm]{geometry}}
\makeatother

%% 控制项目列表
\usepackage{enumerate}
\usepackage{framed}

%% 多栏显示
\usepackage{multicol}

%% 算法环境
\usepackage{algorithm}  
\usepackage{algorithmic} 
\usepackage{float} 

%% 网址引用
\usepackage{url}

%% 控制矩阵行距
\renewcommand\arraystretch{1.4}

%% 粗体
\usepackage{bm}


%% hyperref宏包,生成可定位点击的超链接,并且会生成pdf书签
\makeatletter
\@ifclassloaded{beamer}{
\usepackage{hyperref}
\usepackage{ragged2e} % 对齐
}{
\usepackage[%
    pdfstartview=FitH,%
    CJKbookmarks=true,%
    bookmarks=true,%
    bookmarksnumbered=true,%
    bookmarksopen=true,%
    colorlinks=true,%
    citecolor=blue,%
    linkcolor=blue,%
    anchorcolor=green,%
    urlcolor=blue%
]{hyperref}
}
\makeatother



\makeatletter % 如果是 beamer 不需要下面两个包
\@ifclassloaded{beamer}{
\mode<presentation>
{
} 
}{
%% 控制标题
\usepackage{titlesec}
%% 控制目录
\usepackage{titletoc}
}
\makeatother

%% 控制表格样式
\usepackage{booktabs}

%% 控制字体大小
\usepackage{type1cm}

%% 首行缩进,用\noindent取消某段缩进
\usepackage{indentfirst}

%% 支持彩色文本、底色、文本框等
\usepackage{color,xcolor}

%% AMS LaTeX宏包: http://zzg34b.w3.c361.com/package/maths.htm#amssymb
\usepackage{amsmath,amssymb}
%% 多个图形并排
\usepackage{subfloat}
%%%% 基本插图方法
%% 图形宏包
\usepackage{graphicx}
\newcommand{\red}[1]{\textcolor{red}{#1}}
\newcommand{\blue}[1]{\structure{#1}}
\newcommand{\brown}[1]{\textcolor{brown}{#1}}
\newcommand{\green}[1]{\textcolor{green}{#1}}


%%%% 基本插图方法结束

%%%% pgf/tikz绘图宏包设置
\usepackage{pgf,tikz}
\usetikzlibrary{shapes,automata,snakes,backgrounds,arrows}
\usetikzlibrary{mindmap}
%% 可以直接在latex文档中使用graphviz/dot语言,
%% 也可以用dot2tex工具将dot文件转换成tex文件再include进来
%% \usepackage[shell,pgf,outputdir={docgraphs/}]{dot2texi}
%%%% pgf/tikz设置结束


\makeatletter % 如果是 beamer 不需要下面两个包
\@ifclassloaded{beamer}{

}{
%%%% fancyhdr设置页眉页脚
%% 页眉页脚宏包
\usepackage{fancyhdr}
%% 页眉页脚风格
\pagestyle{plain}
}

%% 有时会出现\headheight too small的warning
\setlength{\headheight}{15pt}

%% 清空当前页眉页脚的默认设置
%\fancyhf{}
%%%% fancyhdr设置结束


\makeatletter % 对 beamer 要重新设置
\@ifclassloaded{beamer}{

}{
%%%% 设置listings宏包用来粘贴源代码
%% 方便粘贴源代码,部分代码高亮功能
\usepackage{listings}

%% 设置listings宏包的一些全局样式
%% 参考http://hi.baidu.com/shawpinlee/blog/item/9ec431cbae28e41cbe09e6e4.html
\lstset{
showstringspaces=false,              %% 设定是否显示代码之间的空格符号
numbers=left,                        %% 在左边显示行号
numberstyle=\tiny,                   %% 设定行号字体的大小
basicstyle=\scriptsize,                    %% 设定字体大小\tiny, \small, \Large等等
keywordstyle=\color{blue!70}, commentstyle=\color{red!50!green!50!blue!50},
                                     %% 关键字高亮
frame=shadowbox,                     %% 给代码加框
rulesepcolor=\color{red!20!green!20!blue!20},
escapechar=`,                        %% 中文逃逸字符,用于中英混排
xleftmargin=2em,xrightmargin=2em, aboveskip=1em,
breaklines,                          %% 这条命令可以让LaTeX自动将长的代码行换行排版
extendedchars=false                  %% 这一条命令可以解决代码跨页时,章节标题,页眉等汉字不显示的问题
}

\usepackage{minted}
\renewcommand{\listingscaption}{Python code} \newminted{python}{
    escapeinside=||,
    mathescape=true,
    numbersep=5pt,
    linenos=true,
    autogobble,
    framesep=3mm} 
}
\makeatother
%%%% listings宏包设置结束


%%%% 附录设置
\makeatletter % 对 beamer 要重新设置
\@ifclassloaded{beamer}{

}{
\usepackage[title,titletoc,header]{appendix}
}
\makeatother
%%%% 附录设置结束





%% 设定行距
\linespread{1}

\newcommand{\bfm}{\boldsymbol m}
\newcommand{\bfc}{\boldsymbol c}
\newcommand{\bfd}{\boldsymbol d}
\newcommand{\bfg}{\boldsymbol g}
\newcommand{\bff}{\boldsymbol f}
\newcommand{\bfx}{\boldsymbol x}
\newcommand{\bfu}{\boldsymbol u}
\newcommand{\bfn}{\boldsymbol n}
\newcommand{\bfv}{\boldsymbol v}
\newcommand{\bft}{\boldsymbol t}
\newcommand{\bfq}{\boldsymbol q}
\newcommand{\bfs}{\boldsymbol s}
\newcommand{\bfy}{\boldsymbol y}
\newcommand{\bfA}{\boldsymbol A}
\newcommand{\bfB}{\boldsymbol B}
\newcommand{\bfC}{\boldsymbol C}
\newcommand{\bfD}{\boldsymbol D}
\newcommand{\bfT}{\boldsymbol T}
\newcommand{\bfP}{\boldsymbol P}
\newcommand{\bfI}{\boldsymbol I}
\newcommand{\bfF}{\boldsymbol F}
\newcommand{\bfK}{\boldsymbol K}
\newcommand{\bfM}{\boldsymbol M}
\newcommand{\bfS}{\boldsymbol S}
\newcommand{\bfW}{\boldsymbol W}
\newcommand{\bfG}{\boldsymbol G}
\newcommand{\bfH}{\boldsymbol H}
\newcommand{\bfQ}{\boldsymbol Q}
\newcommand{\bfJ}{\boldsymbol J}
\newcommand{\balpha}{\bm \alpha}
\newcommand{\blambda}{\bm \lambda}
\newcommand{\bsigma}{\bm \sigma}
\newcommand{\bepsilon}{\bm \epsilon}
\newcommand{\bvarepsilon}{\bm \varepsilon}
\newcommand{\btau}{\bm \tau}
\newcommand{\rmd}{\,\mathrm d}
\newcommand{\cT}{\mathcal T}
\newcommand{\cF}{\mathcal F}
\newcommand{\cS}{\mathcal S}
\newcommand{\cP}{\mathcal P}
\newcommand{\cM}{\mathcal M}
\newcommand{\cA}{\mathcal A}
\newcommand{\cE}{\mathcal E}
\newcommand{\cB}{\mathcal B}
\newcommand{\cQ}{\mathcal Q}
\newcommand{\cN}{\mathcal N}
\newcommand{\cV}{\mathcal V}
\newcommand{\cW}{\mathcal W}
\newcommand{\bbS}{\mathbb S}
\newcommand{\bbR}{\mathbb R}
\newcommand{\od}{\text{div}}
\newcommand{\os}{\text{span}}
\newcommand{\ot}{\text{tr}}
\newcommand{\norm}[1]{||#1||}
\newcommand{\dof}{\text{dof}}

%%%% 个性设置结束
%%%%%%%%------------------------------------------------------------------------


%%%%%%%%------------------------------------------------------------------------
%%%% bibtex设置

%% 设定参考文献显示风格
% 下面是几种常见的样式
% * plain: 按字母的顺序排列,比较次序为作者、年度和标题
% * unsrt: 样式同plain,只是按照引用的先后排序
% * alpha: 用作者名首字母+年份后两位作标号,以字母顺序排序
% * abbrv: 类似plain,将月份全拼改为缩写,更显紧凑
% * apalike: 美国心理学学会期刊样式, 引用样式 [Tailper and Zang, 2006]

%\makeatletter
%\@ifclassloaded{beamer}{
%\bibliographystyle{apalike}
%}{
%\bibliographystyle{abbrv}
%}
%\makeatother


%%%% bibtex设置结束
%%%%%%%%------------------------------------------------------------------------

\input{xecjk_preamble.tex}
\setCJKmainfont{STKaiti} % 如果请替换为本地系统有的字体
%中文断行
\XeTeXlinebreaklocale "zh"
\XeTeXlinebreakskip = 0pt plus 1pt minus 0.1pt
\begin{document}
\title{第一章$\qquad$线性代数的背景}
%\date{\today}
\maketitle  %定义时间和标题
%\tableofcontents
%\newpage
\section{}
\subsection{矩阵}
在这一章中,向量空间是定义在复数域上的,$\mathrm{C}^{n\times m}$ 表示由所有定义在$C$上的$n\times m$的矩阵构成的向量空间。

加法:$C=A+B$,这里的$A$,$B$和$C$都$\in\mathrm{C}^{n\times m}$,即$c_{ij}=a_{ij}+b_{ij}$.

数乘:$C=\alpha A$,即$c_{ij}=\alpha a_{ij}$.

乘法:$C=AB$,这里A$\in\mathrm{C}^{n\times m}$,B$\in\mathrm{C}^{m\times p}$,C$\in\mathrm{C}^{n\times p}$,即$c_{ij}=\sum_{k=1}^m a_{ik}b_{kj}$.

$a_{*j}$表示矩阵A的第j列,$a_{i*}$表示矩阵A的第i行。
$A^H$表示矩阵A的转置共轭矩阵,即$A^H=\bar{A}^H=\bar{A^T}$.
\subsection{方阵与特征值}
I表示单位矩阵,若CA=AC=I,则称矩阵C为矩阵A的逆矩阵$A^{-1}$.

$det(A)=a_{11}A_{11}+a_{11}A_{11}+\cdots+(-1)^{1+n}A_{1n}=\sum_{j=1}^n (-1)^{j+1} a_{1j}det(A_{1j}) $  

如果$det(A)=0$,则称矩阵A是奇异的,否则称A为非奇异的。

$det(AB)=det(A)det(B)$

$det(A^T)=det(A)$

$det(\alpha A)=\alpha ^n det(A)$

$det(\bar{A})=\overline{det(A)}$

$det(I)=1$

\newtheorem{thm}{命题}
\begin{thm}
设A是n阶方阵,如果存在数$\lambda$和非零n维列向量u,使得$Au=\lambda u$成立,则称$\lambda$是A的一个特征值,非零向量u称为矩阵A的对应于特征值$\lambda$的特征向量。

$p_A(\lambda)=det(A-\lambda I)$称为矩阵A的特征多项式,$\lambda $是矩阵A的特征值当且仅当$det(A-\lambda I)\equiv p_A(\lambda)=0$,即$\lambda $是矩阵A的特征多项式的一个根。

矩阵A的谱是指A的所有特征值组成的集合,记为$\sigma (A)$。

矩阵A是奇异的当且仅当A有一个特征值为0。
\end{thm}

\newtheorem{thm}{命题}
\begin{thm}
矩阵A是非奇异的当且仅当A可逆。

矩阵A的谱半径等于矩阵A的特征值的模的最大值,记为$\rho (A)$,矩阵A的迹等于A的所有对角线元素的和,记为$tr(A)=\sum_{i=1}^n a_{ii}$,也等于A的所有特征值的和。
\end{thm}
\newtheorem{thm}{命题}
\begin{thm}
如果$\lambda$是矩阵A的特征值,那么$\bar\lambda$是矩阵$A^H$的特征值。

如果存在一个非零列向量$x$使得$Ax=\lambda x$,则称$\lambda $为一个右特征值,x称为矩阵A的特征值$\lambda $的右特征向量。
如果存在一个非零列向量$y$使得$yA=y\lambda$,则称$\lambda $为一个左特征值,y称为矩阵A的特征值$\lambda $的左特征向量。

$Au=\lambda u$,$v^HA=\lambda v^H$.

$u^HA^H=\bar{\lambda}u^H$,$A^Hv=\bar{\lambda}v$.
\end{thm}

\subsection{矩阵类型}
接下来介绍一些特殊的矩阵类型:

对称矩阵:$A^T=A$

Hermitian矩阵:$A^H=A$

反对称矩阵:$A^T=-A$

反埃尔米特矩阵:$A^H=-A$

正规矩阵:$A^HA=AA^H$

非负矩阵:$a_({j}\ge 0,i,j=1,...,n$(同理定义非正矩阵)

酉矩阵:$Q^HQ=I$

对角矩阵:$a_{ij}=0,i\neq j$,记$A=diag(a_{11},a_{22},...,a_{nn})$

上三角矩阵:$a_{ij}=0,i>j$,类似,下三角矩阵:$a_{ij}=0,i<j$

上双对角线矩阵:$a_{ij}=0,i\neq j$,或者$j\neq i+1$.类似,下双对角矩阵:$a_{ij}=0,i\neq j$,或者$j\neq i-1$.

三对角线矩阵:$a_{ij}=0,|j-i|>1$,形如$A=tridiag(a_{i,i-1},a_{ii},a_{i,i+1})$

带矩阵:$a_{ij}\neq 0,i-m_l\leqslant j\leqslant i+m_u$,$m_l,m_u$是两个非负整数,数$m_l+m_u+1$叫做矩阵的带宽。

Upper Hessenberg 矩阵:$a_{ij}=0,i>j+1$.类似 Lower Hessenberg矩阵:$_{ij}=0,i<j-1$

外积矩阵:$A=uv^H$,此处的uv为列向量。

置换矩阵:它在每行和每列中只有一个1,而在其他地方则为0。

分块对角矩阵:A为n阶方阵,若A的分块矩阵只有在主对角线上有非零子块,其余子块都为零矩阵,且非零子块都是方阵。记$A=diag(A_{11},A_{22},...,A_{nn})$

块三对角矩阵:非零系数在如下的三条对角线上:主对角线、低对角线、高对角线。
\subsection{向量内积和范数}
在一个向量空间X中,内积是一个$X\times X$到R(C)的一个映射:$x\epsilon X,y\epsilon X \longrightarrow s(x,y)\epsilon R $,且它满足如下条件:

1.$s(x,y)$是关于x是线性的,$s(\lambda _1 x_1+\lambda _2 x_2)=\lambda _1 s(x_1,y)+\lambda _2 s(x_2,y),\forall x,y\epsilon X,\forall \lambda _1 ,\lambda _2 \epsilon R(C)$.

2.$s(x,y)=\overline{s(y,x)},\forall x,y\epsilon X$,当$x,y$取实数域时$s(x,y)=s(y,x)$

3.$s(x,y)\ge 0$当且仅当$x=0$时,等号成立。

4.Cauchy-Schwartz不等式:$(\vert s(x,x)\vert)^2\leqslant s(x,x)s(y,y)$.

特别注意当X空间取复数域时,内积$(x,y)=\sum_{i=1}^n x_i\bar{y}_i$,它经常被写成$(x,y)=y^Hx$.欧几里德内积在矩阵计算中的一个基本性质:$(Ax,y)=(x,Ay),\forall x,y\epsilon C$

\newtheorem{thm}{命题}
\begin{thm}
酉矩阵保留欧几里德内积,$(Qx,Qy)=(x,y)$,对任意酉矩阵和向量$x,y$都成立。
\end{thm}

证明:$(Qx,Qy)=(x,Q^HQ)=(x,y)$

在内积空间中,我们可以定义范数,令$\parallel x\parallel=\sqrt{(x,x)}$,则此由内积定义的范数满足:
1.$\parallel x\parallel \ge 0,\forall x\epsilon X$,并且$\parallel x\parallel=0$当且仅当$x=0$.

2.$\parallel \alpha x\parallel =|\alpha| \parallel x\parallel,\forall x\epsilon X,\forall \alpha \epsilon C$.

3.$\parallel x+y \parallel \leqslant \parallel x\parallel + \parallel y\parallel ,\forall x,y\epsilon X$.

从上面的命题可以看出,一个酉矩阵保留欧几里德范数,$\parallel Qx \parallel_2=\parallel x \parallel_2,\forall x$.由此我们可以看出酉矩阵是保范的。下面介绍向量的几个范数:
$\parallel x \parallel_p=(\sum_{i=1}^n |x_i|^p)^{1/p}$.且当$p=1,2,\infty$时

$\parallel x \parallel_1=|x_1|+|x_2|+...+|x_n|$

$\parallel x \parallel_2=[|x_1|^2+|x_2|^2+..+|x_n|^2]^{1/2}$

$\parallel x \parallel_\infty=\max |x_i|$

Cauchy-Schwartz不等式变成$|(x,y)|\leqslant \parallel x \parallel_2\parallel y\parallel_2$.





















\subsection{正交向量和子空间}
一个向量集合$G={a_1,a_2,...a_r}$如果满足$(a_i,a_j)=0,i\neq j$,则称为正交集。且对$G$的每一个向量$(\parallel a_i \parallel)_2=1$,称为标准正交集。一个向量正交于子空间$S$的每一个向量,就称正交于这个子空间。与$S$所有正交的向量的构成的集合是一个子空间,称为$S$的正交补,记$S^\perp$。且任何向量都可以分解成$S$与$S^\perp$中的向量和。

每一个子空间都有一个标准正交基,这个标准正交基是通过任意一组基标准正交化得到的,通常用Gram-Schmidt算法求出。

\begin{definition}
设$A\epsilon C^{n\times n}$,如果存在$n$阶酉矩阵$Q$和$n$阶上三角矩阵$R$,使$A=QR$则称之为$A$的$QR$分解或酉三角分解。当$A\epsilon R^{n\times n}$时,称为$A$的正三角分解。
\end{definition}

任意满秩实(复)矩阵$A$,都可以唯一地分解$A=QR$,其中$Q$为正交(酉)矩阵,$R$是具有正对角元的上三角矩阵。
\subsection{矩阵的规范形式}
下面讨论将方阵简化对角线形式,或者上三角形式的简单方阵。且这种简单转化保留特征值不变。

\begin{definition}
矩阵$A,B$称为相似的,如果存在一个非奇异矩阵$X$使得   $A=X^{-1}BX$  
\end{definition}

对进行运算称为对进行相似变换,称可逆矩阵为相似变换矩阵。

下面介绍一些术语:

1.矩阵$A$的一个特征值的有几重根,这个几重根就叫做这个特征值的代数重数。

2.如果一个特征值的代数重数是1,则这个特征值是简单的;矩阵$A$的特征值代数重数不是1,则是非简单的。

3.矩阵$A$的特征值所对应的特征向量所构成空间(即特征子空间,也是方程组$(λI-A)x=0$的维数,称为几何重数。

4.如果$A$的某个特征值的几何重数严格小于代数重数,我们就说$A$是有亏的;如果$A$的每个特征值的几何重数都与它的代数重数相等,我们就称$A$是无亏的;如果$A$的每个特征值的几何重数都是1,我们称A是无损的,否则有损的。
\subsubsection{矩阵可对角化}
\newtheorem{thm}{定理}
\begin{thm}
$n$维矩阵可对角化,当且仅当它有$n$个线性无关的特征向量。
\end{thm}

\newtheorem{thm}{定理}
\begin{thm}
$n$维矩阵可对角化,当且仅当它的所有特征值是半单特征值。(半单特征值指特征值无重根)
\end{thm}

\subsubsection{若尔当标准型}
\begin{definition}
若尔当标准型是由若干个主对角线为特征值,下方(或上方)次对角线全为1,其余全为0的若尔当块按对角排列组成的准对角矩阵。不是每个n阶矩阵通过初等变换都能化为对角矩阵,但每个n阶复数矩阵A通过初等变换都能化为若尔当标准型,这个若尔当形矩阵除去其中若尔当块的排列次序不同外是被矩阵A唯一确定的,它称为矩阵A的若尔当标准型。
\end{definition}

\newtheorem{thm}{定理}
\begin{thm}
如何一个矩阵$A$都可以简化为由$p$个对角块组成的对角矩阵,每个对角块与一个不同特征值$\lambda _i$有关。每个对角块本身是由$\gamma _i$个子块构成的对角结构,其中$\gamma _i$是特征值$\lambda _i$的几何重数。每个子块是若尔当块,由这些若尔当块组成的准对角矩阵称为若尔当标准型。

一个矩阵$A$存在一个可逆矩阵$X$,使得$X^{-1}AX=J=\begin{bmatrix}
J_1 &\cdots & 0\\
0 &J_2\cdots & 0\\
\vdots & \ddots & \vdots\\
0 & \cdots & J_p
\end{bmatrix}$,

$\begin{bmatrix}
J_{i1} &\cdots & 0\\
0 &J_{i2}\cdots & 0\\
\vdots & \ddots & \vdots\\
0 & \cdots & J_{i\gamma _i}
\end{bmatrix}$,其中
$\begin{bmatrix}
\gamma _i &1&\cdots & 0&0\\
0 &\gamma _i&1&\cdots & 0\\
0&0&\vdots & \ddots & \vdots\\
0&0&\vdots & \ddots &1\\
0 &0&0& \cdots & \gamma _i
\end{bmatrix}$

每一个$J_{ik}$块对应一个与特征值$\gamma _i$有关的特征向量。
\end{thm}
\subsection{Schur分解}
下面将得出任何一个方阵都可以唯一相似一个上三角矩阵。

\newtheorem{thm}{定理}
\begin{thm}
对任意方阵$A$,存在一个酉矩阵$Q$,使得$Q^HAQ=R$,其中$R$是一个上三角矩阵。
\end{thm}

这个其实就是Schur分解,当$AQ=QR$时,推出对任意$i\leqslant j\leqslant k$,我们有$Aq_j=\sum_{i=1}^{i=j} r_{ij}q_i$.如果将矩阵$Q,R$按列分块,令$Q_k=[q_1,q_2,...,q_k]$,并且$R_k$也是$R$的前$k$列,我们同样有$AQ_k=Q_kR_k$,这被称为矩阵$A$的部分舒尔分解。并且最简单的形式当$k=1$时,这里的$q_1$是一个特征向量。平时$q_i$被称为舒尔向量。
\subsection{矩阵幂的应用}
\newtheorem{thm}{定理}
\begin{thm}
序列$A^k,k=0,1,...,$收敛于0时当且仅当$\rho (A)<1$。
\end{thm}

\newtheorem{thm}{定理}
\begin{thm}
级数$\sum_{k=0}^\infty A^k$ 收敛当且仅当$\rho (A)<1$。在这个条件下,$I-A$是非奇异的并且这个级数的极限等于$(I-A)^{-1}$。
\end{thm}

\newtheorem{thm}{定理}
\begin{thm}
对于任意矩阵范数$\parallel .\parallel$,我们有$\lim_{k \to \infty}\parallel A^k\parallel ^{1/k}=\rho (A)$
\end{thm}
















\subsection{正定矩阵}
一个实方阵是正定的如果$(Au,u)>0$,$\forall$ $ u  \epsilon  R^n,u\neq 0.$

任一个方阵$A$(实或者复)都可以分解成
$$A=H+iS,$$
其中
$$H=\frac{1}{2}(A+A^H),$$
$$S=\frac{1}{2i}(A-A^H).$$

则矩阵$H,S$是Hermitian,而$iS$是skew-Hermitian.这个分解类似与一个数$z$可以分解成$z=x+iy$,
$$x=\Re (z)=\frac{1}{2}(z+\bar{z}),y=\Im (z)=\frac{1}{2i}(z-\bar{z}).$$

当方阵$A$是实的,$u$是实向量时,$(Au,u)$是实的,结果上述的分解$A=H+iS$可以写成$(Au,u)=(Hu,u)$,由此可以得到下面定理

\begin{thm}
令方阵$A$是实的正定矩阵,则$A$是非奇异的。此外,存在一个系数$\alpha >0$使得
$$(Au,u)\geqslant \alpha \parallel x \parallel _2^2,\forall u\epsilon R^n$$
\end{thm}

\begin{thm}
当$A$是任意方阵(可能为复方阵),令$H=\frac{1}{2}(A+A^H),S=\frac{1}{2i}(A-A^H)$。则矩阵$A$的任意特征向量$\lambda _i$使得
$$\lambda_{min}(H)\leqslant \Re (\lambda _j)\leqslant \lambda_{max}(H),$$
$$\lambda_{min}(S)\leqslant \Im (\lambda _j)\leqslant \lambda_{max}(S).$$
\end{thm}

\subsection{投影算子}
投影算子是在赋范线性空间$X$上具有幂等性的有界线性算子。设$P$是$X$上的有界线性算子,如果$P^2=P$,则称P为投影算子。投影算子有一些简单性质$Null(P)=Ran(I-P)$,此外,$Null(P)\bigcap Ran(P)=\emptyset$。并且$C^n$中每一个元素都可以被写成$x=Px+(I-P)x$,因此空间$C^n$可以被直和分解成$C^n=Null(P)+Ran(P)$
\newtheorem{thm}{定理}
\begin{thm}
给定$C^n$空间的两个相同维数的子空间$M$和$L$,下面的两个条件是相互等价的。

(1).子空间$M$中没用非零向量垂直于$L$;

(2).对于$\forall x\epsilon C^n$,存在唯一的$u$,使得$u\epsilon M,x-u\epsilon L^\bot$
\end{thm}

\subsection{正交投影算子}
\begin{definition}
设$H$是希尔伯特空间,$M$是$H$上的闭子空间。对$\forall x\epsilon H$,令$Px=x_0$为$x$在$M$上的正交投影,即$x=Px+y$,其中 $Px\epsilon M,y\epsilon M^\perp$。称$P$为$H$到$M$上正交投影算子。
\end{definition}

正交投影算子具有以下几个基本性质:

1.正交投影算子$P$是有界线性算子,且$\parallel P \parallel=1$。

2.投影算子$P$是幂等算子,即$P^n=P$

3.$P$是一个投影算子,则$I-P$也是一个投影算子,并且满足$Null(P)=Ran(I-P)$

$P$是一个正交投影算子,则向量$Px,(I-P)x$是正交的,且满足$\parallel x\parallel _2^2=\parallel Px\parallel _2^2+\parallel (I-P)x\parallel _2^2$,由此可知$\parallel Px\parallel _2\leqslant \parallel x\parallel _2$,所以$\max \parallel Px\parallel _2 /\parallel x\parallel _2\leqslant 1$,即$\parallel P\parallel _2\leqslant 1$。

\begin{thm}
$P$是子空间$M$上的正交算子,则$C^n$上的任意向量$x$,和$M$中任意向量$y$都满足
$$\min \parallel x-y\parallel _2\leqslant \arallel x-Px\parallel _2$$
\end{thm}

并且任给定子空间$M$,和$C^n$中任意向量$x$,则$\min \limits_{y\epsilon M} \parallel x-y\parallel _2= parallel x-x^*\parallel _2$,当且仅当它满足
$$\begin{cases}
y^*  \epsilon M \\
x-y^*\perp M
\end{cases}
$$










\title{第二章$\qquad$偏微分方程离散化}
%\date{\today}
\maketitle  %定义时间和标题
%\tableofcontents
%\newpage
\section{}
\subsection{偏微分方程}
偏微分方程是稀疏矩阵问题的最大来源。求解这类方程的典型方法是离散化,例如,用有限个未知数的方程来近似它们。这些离散化产生的矩阵问题通常是大而稀疏的,例如,它们的非零分量很少,有几种不同的方法可以离散偏微分方程。最简单的方法是对偏微分算子采用有限差分近似。有限元法用一个在全局域上具有一定导数的函数代替了原来的函数,在简单的单元上是分段多项式,如小三角形或矩形。这种方法可能是目前最通用和最容易理解的离散化技术。在这两种方法之间,还有一个被称为有限体积法的保守方法,它们试图模拟连续的物理守恒定律。本章介绍了这三种不同的离散方法。

\subsubsection{椭圆算子}
Poisson方程:                   

$$\dfrac{\partial ^2u}{\partial x_1^2}+\dfrac{\partial ^2u}{\partial x_2^2}=f$,对于$\forall x=\begin{pmatrix}
x_1  \\
x_2
\end{pmatrix} \epsilon  \Omega$$

$\Omega$是$R^n$上的一个有界的开集,$x_1,x_2$是两空间变量。上述方程只适用于位于域$\Omega$内部的点,同样重要的是在$\Omega$的边界$\mathcal{T}$上必须满足的条件。这些被称为边界条件,它们有三种常见类型:

Dirichlet条件:           $u(x)=\phi (x)$

Neumann条件:             $\frac{\partial u}{\partial \vec{n}}(x)=0$

Cauchy条件:$\frac{\partial u}{\partial \vec{n}}(x)+\alpha(x)u(x)=\gamma (x)$

 $\vec n$ 是 $\Omega$ 边界 $\mathcal{T}$ 上的单位外法线向量.对于给定的单位向量$\vec{v}$有两分量$v_1,v_2$,方向导数$\frac{\partial u}{\partial \vec{v}}$定义为:
 
 $$\frac{\partial u}{\partial \vec{v}}(x)=\lim_{n \to\ 0}\frac{u(x+h\vec{v})-u(x)}{h}$

 $=\frac{\partial u}{\partial x_1}(x)v_1+\frac{\partial u}{\partial x_2}(x)v_2$

 $=\triangledown u.\vec{v}$$

 $\triangledown$为$u$的梯度:
 $$\triangledown u=\begin{pmatrix}
 \frac{\partial u}{\partial x_1} \\
\frac{\partial u}{\partial x_2}
\end{pmatrix}$$

特殊情况是当$f(x)=0$时,$\triangledown u=0$,并且边界条件必须加上,这就是拉普拉斯方程。

拉普拉斯算子:
$$\triangle=\dfrac{\partial ^2}{\partial x_1^2}+\dfrac{\partial ^2}{\partial x_2^2}$$

椭圆算子:
$$L=\dfrac{\partial}{\partial x_1}(a\dfrac{\partial}{\partial x_1})+\dfrac{\partial}{\partial x_2}(a\dfrac{\partial}{\partial x_2})=\triangledown .(a\triangledown)$$

散度:
$$div\vec{v}=\triangledown .\vec{v}=\dfrac{\partial v_1}{\partial x_1}+\dfrac{\partial v_2}{\partial x_2}$$

由此椭圆算子可以写成:
$$L=\dfrac{\partial}{\partial x_1}(a_1\dfrac{\partial}{\partial x_1})+\dfrac{\partial}{\partial x_2}(a_2\dfrac{\partial}{\partial x_2})=\triangledown (\vec .a\triangledown)$$

\subsubsection{对流扩散方程}
许多问题涉及“对流”和“扩散”的结合。用对流扩散方程模拟了下面的现象:

$$\dfrac{\partial u}{\partial t}+b_1\dfrac{\partial u}{\partial x_1}+b_2\dfrac{\partial u}{\partial x_2}=\triangledown .(a\triangledown)u+f$$
或者
$$\dfrac{\partial u}{\partial t}+\vec b.\triangledown u=\triangledown .(a\triangledown)u+f$$

这个稳定状态的版本还可以写成:$$-\triangledown .(a\triangledown)u+\vec b.\triangledown u=f$$


%\end{document}



 





















































%\newtheorem{thm}{定理}
%\begin{thm}

%\end{thm}



































%函数$y(x)$的变分定义为$\delta y=y_1(x)-y(x)$,其中$y_1(x)$是“靠近”$y(x)$的一个函数,即$\delta y$是同一自变量$x$处相邻函数的函数值之差。

%注意:

%$(\delta y)'=y'_1(x)-y'(x)=\delta y'$

%$(\delta y)^n=\delta y^n $
%\subsection{泛函的变分}
%定义泛函$J[y(x)]=\int_{a}^{b} f(x,y,y')\mathrm{d}x$,则增量$\bigtriangleup J=\int_{a}^{b}[f(x,y+\delta y,y'+\delta y')-f(x,y,y')]\mathrm{d}x=\int_{a}^{b}[\frac{\partial f}{\partial y}\delta y + \frac{\partial f}{\partial y'}\delta y'+\frac{1}{2}\frac{\partial^2 f}{\partial^2 y}(\delta y)^2+\frac{\partial^2 f}{\partial y\,\partial y'}\delta y\delta y'+\frac{1}{2}\frac{\partial^2 f}{\partial y'^2}(\partial y')^2+\cdots]\mathrm{d}x$

%舍弃掉$\delta y$和$\delta y'$二次项及以上的高次项,得到关于$\delta y$和$\delta y'$一次项的和,称为$J[y(X)]=\int_{a}^{b} f(x,y,y')\mathrm{d}x$的一阶变分,简称为泛函的变分,即$\delta J=\int_{a}^{b}(\frac{\partial f}{\partial y}\delta y + \frac{\partial f}{\partial y'}\delta y')\mathrm{d}x$。
%\subsection{泛函变分的基本运算法则}
%泛函变分运算法则与微分运算法则基本相同

%$(\delta F_1 +F_2)=\delta F_1 +\delta F_2$

%$(\delta F)^n=nF^n-1\delta F$

%$(\delta F_1 \cdot F_2)=F_2\delta F_1+F_1\delta F_2$

%$(\delta (\frac{F_1}{F_2})=\frac{1}{F^2_2}(F_2\delta F_1-F_1\delta F_2)$

%$\delta\int_{a}^{b}F\mathrm{d}x=\int_{a}^{b}\delta F\mathrm{d}x$







%如果将泛函取极值时的函数定义为$y(x)$,并且定义与函数$y(x)$相靠近的函数为$y(x,\varepsilon)$,记为$y(x,\varepsilon)=y(x)+\varepsilon\eta(x)$,$\varepsilon$是一个参数。
%函数$y(x)$的变分定义为$\delta y=\eta(x)\mathrm{d}\varepsilon$,由此可得$\delta y'=\eta ^\prime(x)\mathrm{d}\varepsilon$
%定义泛函$J[y(X)]=\int_{a}^{b} F(x,y,y')\mathrm{d}x$的变分为$\delta J=\int_{a}^{b}(\frac{\partial F}{\partial y}\delta y + \frac{\partial F}{\partial y'}\delta y')\mathrm{d}x$。

%\subsection{泛函变分举例}
%计算泛函$J[y(x)]=\int_{-1}^{1} (y'e^7 +xy^2)\mathrm{d}x$的变分


%$\delta J[y(x)]=\delta\int_{-1}^{1}(y'e^7+xy^2)\mathrm{d}x=\int_{-1}^{1}(2xy\delta y+e^7\delta y')\mathrm{d}x=\int_{-1}^{1}(2xy\delta y)\mathrm{d}x+\int_{-1}^{1}e^7\mathrm{d}\delta y=\int_{-1}^{1}(2xy\delta y)\mathrm{d}x
%$,最后一步利用上边界上函数变分为0。






%\begin{equation}
%-\nabla \cdot (\beta\nabla u) = f(x,y),\,\ (x,y)\in \Omega
%\end{equation}
%Dilichlet 边界条件

%\begin{equation}
%u(x,y) = g(x,y),\,\ (x,y)\in \partial \Omega
%\end{equation}
%\subsection{符号}
%\begin{tabular}{ |l|l| }   
%\hline   
%\multicolumn{2}{|c|}{符号说明} \\   
%\hline
%符号 & 含义\\
%\hline
%$\Omega$ & 二维长方形区域 \\
%\hline
%$nx$ & $x$ 方向剖分的段数 \\
%\hline
%$ny$ & $y$ 方向剖分的段数 \\
%\hline
%$hx$ &  $x$ 方向每段的长度\\
%\hline
%$hy$ &  $y$ 方向每段的长度 \\
%\hline
%$\mu$ & $the \,\ viscosity \,\ coefficient$ \\
%\hline
%$k$ & $the \,\ permeability \,\ tensor$ \\
%\hline 
%$NC$ & 代表 $cell$ 的个数 \\
%\hline
%$NE$ & 代表总的 $edge$ 的个数 \\
%\hline
%\end{tabular}

%\section{欧拉—拉格朗日方程}
%欧拉—拉格朗日方程是泛函极值问题的必要条件,假设$J[y(x)]$的极值问题的解为$y=y(x)$,现在推导这个解所满足的微分方程。

%$\delta J=\int_{a}^{b}(\frac{\partial f}{\partial y}\delta y + \frac{\partial f}{\partial y'}\delta y')\mathrm{d}x=0$,将第二项分部积分得到$\int_{a}^{b}(\frac{\partial f}{\partial y'}\delta y')\mathrm{d}x=\int_{a}^{b}\frac{\partial f}{\partial y'}\mathrm{d}\delta y$,因为$\delta y(a)=0$,$\delta y(b)=0$,所以$\int_{a}^{b}(\frac{\partial f}{\partial y'}\delta y')\mathrm{d}x=-\int_{a}^{b}\delta y\mathrm{d}\frac{\partial f}{\partial y'}$,因此$\delta J=\int_{a}^{b}\frac{\partial f}{\partial y}\delta y\mathrm{d}x-\int_{a}^{b}\delta y\mathrm{d}\frac{\partial f}{\partial y'}=\int_{a}^{b}(\frac{\partial f}{\partial y}-\frac{\mathrm{d}\frac{\partial f}{\partial y'}}{\mathrm{d}x})\delta y\mathrm{d}x=0$,因为对于任何函数$\delta y$都成立,故$\frac{\partial f}{\partial y}-\frac{\mathrm{d}\frac{\partial f}{\partial y'}}{\mathrm{d}x}=0$,这就是欧拉—拉格朗日方程。
%\begin{equation*}
%\begin{cases}
%\begin{aligned}
%\frac{\mu}{k}\mathbf{u} + \nabla p & = 0 \quad in \,\ \Omega = (0,1)\times (0,1) \\
%\nabla \cdot \mathbf{u} & = f \quad in \,\ \Omega \\
%\mathbf {u} & = 0 \quad on \,\ \partial \Omega
%\end{aligned}
%\end{cases}
%\end{equation*}

%且有 \\
%\begin{equation*}
%\int_{\Omega}f dxdy = 0
%\end{equation*}

%记 $u$ 为 $\mathbf{u}$ 在 $x$ 方向的分量,$v$ 为 $\mathbf{u}$ 在 $y$ 方向的分量,则有 \\

%\begin{equation*}
%\begin{cases}
%\begin{aligned}
%\frac{\mu}{k}\cdot u + \partial_x p & = 0 \quad (1) \\
%\frac{\mu}{k}\cdot v + \partial_y p & = 0 \quad (2) \\
%\partial_x u + \partial_y v & = f \quad (3)
%\end{aligned}
%\end{cases}
%\end{equation*}

%\section{离散后组装矩阵}
%利用一阶向前差分把方程变成差分方程,现在从 $edge$ 和 $cell$ 的角度考虑模型。 \\

%对于 $(1)$, 从内部纵向 $edge$ 的角度考虑:
%我们需要找到内部纵向 $edge$ 所对应的左手边的 $cell$ 和右手边的 $cell$. 左右两边的$cell$ 所对应的 $p$ 分别记为 $p_{l}$、$p_{r}$.$u$ 为 $edge$ 的中点,记为 $u_m$。按照 $mesh$ 里的编号规则排序。\\

%则每条内部边上所对应的差分方程为:

%\begin{equation*}
%\frac{\mu}{k} \cdot u_m + \frac{p_r - p_l}{hx} = 0
%\end{equation*}

%对于 $(2)$,从内部横向 $edge$ 的角度考虑:
%我们需要找到内部横向 $edge$ 所对应的左手边的 $cell$ 和右手边的 $cell$. $cell$ 所对应的 $p$ 与 $(1)$ 中的相同。$v$ 为 $edge$ 的中点,记为 $v_m$。\\

%则每条内部边上所对应的差分方程为:\\

%\begin{equation*}
%\frac{\mu}{k} \cdot v_m + \frac{p_l - p_r}{hy} = 0
%\end{equation*}

%对于 $(3)$, 从 $cell$ 的角度考虑:
%由于单元是四边形单元,我们记单元所对应边的局部编号为[0,1,2,3](StructureQuadMesh.py 里的网格),第 $i$ 个单元所对应的边记为 $e_{i,0},e_{i,1},e_{i,2},e_{i,3}$。\\

%则 $(3)$ 式第 $i$ 个单元所对应的差分方程为:\\

%\begin{equation*}
%\frac{u_{e_{i,1}} - u_{e_{i,3}}}{hx} + \frac{v_{e_{i,2}} - v_{e_{i,0}}}{hy} = f_i
%\end{equation*}

%我们需要生成一个 $(NE+NC)\times(NE+NC)$的系数矩阵,把它看成分块矩阵
%\begin{equation*}
%\begin{pmatrix}
% A_{1,1} & A_{1,2} \\
%A_{2,1} & A_{2,2}
%\end{pmatrix}
%\end{equation*}

%其中 \\

%\begin{equation*}
%\begin{aligned}
%A_{1,1} : NE\times NE \\
%A_{1,2} : NE\times NC \\
%A_{2,1} : NC\times NE \\
%A_{2,2} : NC\times NC
%\end{aligned}
%\end{equation*}

%$A_{1,1}$ 对应的是 $(1),(2)$ 两式的第一项,即含有 $u,v$ 的项,$A_12$ 对应的是 $(1),(2)$ 两式的第二项。

%\newpage
%\nocite{*}
%\bibliography{ref}
\end{document}

