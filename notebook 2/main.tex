% !Mode:: "TeX:UTF-8"
\documentclass[12pt,a4paper]{article}

%%%%%%%%------------------------------------------------------------------------
%%%% 日常所用宏包

%% 控制页边距
% 如果是beamer文档类, 则不用geometry
\makeatletter
\@ifclassloaded{beamer}{}{\usepackage[top=2.5cm, bottom=2.5cm, left=2.5cm, right=2.5cm]{geometry}}
\makeatother

%% 控制项目列表
\usepackage{enumerate}
\usepackage{framed}

%% 多栏显示
\usepackage{multicol}

%% 算法环境
\usepackage{algorithm}  
\usepackage{algorithmic} 
\usepackage{float} 

%% 网址引用
\usepackage{url}

%% 控制矩阵行距
\renewcommand\arraystretch{1.4}

%% 粗体
\usepackage{bm}


%% hyperref宏包,生成可定位点击的超链接,并且会生成pdf书签
\makeatletter
\@ifclassloaded{beamer}{
\usepackage{hyperref}
\usepackage{ragged2e} % 对齐
}{
\usepackage[%
    pdfstartview=FitH,%
    CJKbookmarks=true,%
    bookmarks=true,%
    bookmarksnumbered=true,%
    bookmarksopen=true,%
    colorlinks=true,%
    citecolor=blue,%
    linkcolor=blue,%
    anchorcolor=green,%
    urlcolor=blue%
]{hyperref}
}
\makeatother



\makeatletter % 如果是 beamer 不需要下面两个包
\@ifclassloaded{beamer}{
\mode<presentation>
{
} 
}{
%% 控制标题
\usepackage{titlesec}
%% 控制目录
\usepackage{titletoc}
}
\makeatother

%% 控制表格样式
\usepackage{booktabs}

%% 控制字体大小
\usepackage{type1cm}

%% 首行缩进,用\noindent取消某段缩进
\usepackage{indentfirst}

%% 支持彩色文本、底色、文本框等
\usepackage{color,xcolor}

%% AMS LaTeX宏包: http://zzg34b.w3.c361.com/package/maths.htm#amssymb
\usepackage{amsmath,amssymb}
%% 多个图形并排
\usepackage{subfloat}
%%%% 基本插图方法
%% 图形宏包
\usepackage{graphicx}
\newcommand{\red}[1]{\textcolor{red}{#1}}
\newcommand{\blue}[1]{\structure{#1}}
\newcommand{\brown}[1]{\textcolor{brown}{#1}}
\newcommand{\green}[1]{\textcolor{green}{#1}}


%%%% 基本插图方法结束

%%%% pgf/tikz绘图宏包设置
\usepackage{pgf,tikz}
\usetikzlibrary{shapes,automata,snakes,backgrounds,arrows}
\usetikzlibrary{mindmap}
%% 可以直接在latex文档中使用graphviz/dot语言,
%% 也可以用dot2tex工具将dot文件转换成tex文件再include进来
%% \usepackage[shell,pgf,outputdir={docgraphs/}]{dot2texi}
%%%% pgf/tikz设置结束


\makeatletter % 如果是 beamer 不需要下面两个包
\@ifclassloaded{beamer}{

}{
%%%% fancyhdr设置页眉页脚
%% 页眉页脚宏包
\usepackage{fancyhdr}
%% 页眉页脚风格
\pagestyle{plain}
}

%% 有时会出现\headheight too small的warning
\setlength{\headheight}{15pt}

%% 清空当前页眉页脚的默认设置
%\fancyhf{}
%%%% fancyhdr设置结束


\makeatletter % 对 beamer 要重新设置
\@ifclassloaded{beamer}{

}{
%%%% 设置listings宏包用来粘贴源代码
%% 方便粘贴源代码,部分代码高亮功能
\usepackage{listings}

%% 设置listings宏包的一些全局样式
%% 参考http://hi.baidu.com/shawpinlee/blog/item/9ec431cbae28e41cbe09e6e4.html
\lstset{
showstringspaces=false,              %% 设定是否显示代码之间的空格符号
numbers=left,                        %% 在左边显示行号
numberstyle=\tiny,                   %% 设定行号字体的大小
basicstyle=\scriptsize,                    %% 设定字体大小\tiny, \small, \Large等等
keywordstyle=\color{blue!70}, commentstyle=\color{red!50!green!50!blue!50},
                                     %% 关键字高亮
frame=shadowbox,                     %% 给代码加框
rulesepcolor=\color{red!20!green!20!blue!20},
escapechar=`,                        %% 中文逃逸字符,用于中英混排
xleftmargin=2em,xrightmargin=2em, aboveskip=1em,
breaklines,                          %% 这条命令可以让LaTeX自动将长的代码行换行排版
extendedchars=false                  %% 这一条命令可以解决代码跨页时,章节标题,页眉等汉字不显示的问题
}

\usepackage{minted}
\renewcommand{\listingscaption}{Python code} \newminted{python}{
    escapeinside=||,
    mathescape=true,
    numbersep=5pt,
    linenos=true,
    autogobble,
    framesep=3mm} 
}
\makeatother
%%%% listings宏包设置结束


%%%% 附录设置
\makeatletter % 对 beamer 要重新设置
\@ifclassloaded{beamer}{

}{
\usepackage[title,titletoc,header]{appendix}
}
\makeatother
%%%% 附录设置结束





%% 设定行距
\linespread{1}

\newcommand{\bfm}{\boldsymbol m}
\newcommand{\bfc}{\boldsymbol c}
\newcommand{\bfd}{\boldsymbol d}
\newcommand{\bfg}{\boldsymbol g}
\newcommand{\bff}{\boldsymbol f}
\newcommand{\bfx}{\boldsymbol x}
\newcommand{\bfu}{\boldsymbol u}
\newcommand{\bfn}{\boldsymbol n}
\newcommand{\bfv}{\boldsymbol v}
\newcommand{\bft}{\boldsymbol t}
\newcommand{\bfq}{\boldsymbol q}
\newcommand{\bfs}{\boldsymbol s}
\newcommand{\bfy}{\boldsymbol y}
\newcommand{\bfA}{\boldsymbol A}
\newcommand{\bfB}{\boldsymbol B}
\newcommand{\bfC}{\boldsymbol C}
\newcommand{\bfD}{\boldsymbol D}
\newcommand{\bfT}{\boldsymbol T}
\newcommand{\bfP}{\boldsymbol P}
\newcommand{\bfI}{\boldsymbol I}
\newcommand{\bfF}{\boldsymbol F}
\newcommand{\bfK}{\boldsymbol K}
\newcommand{\bfM}{\boldsymbol M}
\newcommand{\bfS}{\boldsymbol S}
\newcommand{\bfW}{\boldsymbol W}
\newcommand{\bfG}{\boldsymbol G}
\newcommand{\bfH}{\boldsymbol H}
\newcommand{\bfQ}{\boldsymbol Q}
\newcommand{\bfJ}{\boldsymbol J}
\newcommand{\balpha}{\bm \alpha}
\newcommand{\blambda}{\bm \lambda}
\newcommand{\bsigma}{\bm \sigma}
\newcommand{\bepsilon}{\bm \epsilon}
\newcommand{\bvarepsilon}{\bm \varepsilon}
\newcommand{\btau}{\bm \tau}
\newcommand{\rmd}{\,\mathrm d}
\newcommand{\cT}{\mathcal T}
\newcommand{\cF}{\mathcal F}
\newcommand{\cS}{\mathcal S}
\newcommand{\cP}{\mathcal P}
\newcommand{\cM}{\mathcal M}
\newcommand{\cA}{\mathcal A}
\newcommand{\cE}{\mathcal E}
\newcommand{\cB}{\mathcal B}
\newcommand{\cQ}{\mathcal Q}
\newcommand{\cN}{\mathcal N}
\newcommand{\cV}{\mathcal V}
\newcommand{\cW}{\mathcal W}
\newcommand{\bbS}{\mathbb S}
\newcommand{\bbR}{\mathbb R}
\newcommand{\od}{\text{div}}
\newcommand{\os}{\text{span}}
\newcommand{\ot}{\text{tr}}
\newcommand{\norm}[1]{||#1||}
\newcommand{\dof}{\text{dof}}

%%%% 个性设置结束
%%%%%%%%------------------------------------------------------------------------


%%%%%%%%------------------------------------------------------------------------
%%%% bibtex设置

%% 设定参考文献显示风格
% 下面是几种常见的样式
% * plain: 按字母的顺序排列,比较次序为作者、年度和标题
% * unsrt: 样式同plain,只是按照引用的先后排序
% * alpha: 用作者名首字母+年份后两位作标号,以字母顺序排序
% * abbrv: 类似plain,将月份全拼改为缩写,更显紧凑
% * apalike: 美国心理学学会期刊样式, 引用样式 [Tailper and Zang, 2006]

%\makeatletter
%\@ifclassloaded{beamer}{
%\bibliographystyle{apalike}
%}{
%\bibliographystyle{abbrv}
%}
%\makeatother


%%%% bibtex设置结束
%%%%%%%%------------------------------------------------------------------------

\input{../xecjk_preamble.tex}
\numberwithin{equation}{section}
\renewcommand {\thetable} {\thesection{}.\arabic{table}}
\renewcommand {\thefigure} {\thesection{}.\arabic{figure}}
\title{The Shapes of Things}
\author{龚欣}
\date{\chntoday}

\begin{document}
\maketitle

\section{第一章~~~~~~介绍}
\subsection{曲面上微分方程}
这本书是对微分几何的介绍,这对含有几何偏微分方程(PDE)的数学模型很有用,例如标准拉普拉斯方程的曲面或流形(manifold)。尤其是,这需要发展曲面梯度和曲面拉普拉斯算子。这些在标准欧几里得空间里(如$\mathbb{R}^n$)是普通的梯度$\nabla$和拉普拉斯算子$\Delta =\nabla \cdot \nabla$,但是在曲面或流体上定义就会变得特殊。\\

这种方法的优点是,它提供了几何量的替代公式,例如这些在微分几何中的公式,和(平均)曲率(curvature)比文章平时的表达式更清楚。

\subsection{关于形状的微分}
这本书中的微分几何方法有利于形状微分学的框架发展(framework of shape differential calculus),这是研究一个独立的“形状变量”的如何随其变化而变化的。

\subsubsection{一个简单例子}
接下来的例子需要freshman calculus的工具。令$f=f(r,\theta)$是定义在极坐标下半径为$R$的圆盘$\Omega$上,$F$是圆盘上的积分,如
\begin{gather}
F=\int_{\Omega} f=\int_{0}^{2\pi}\int_{0}^{R} f(r,\theta)r\mathrm{d}r\mathrm{d}\theta.
\end{gather}

显然,$F$依赖$R$,我们假设$f$总是依赖$R$,如$f=f(r,\theta;R)$。在物理例子中,$F$是液体通过横截面积为$\Omega$的管道的净流量(net flow rate)。在这种情况下,$f$是单位面积下的流速,可以是定义在$\Omega$上的偏微分方程的解。例如,在圆管中的Navier-Stokes流体。

知道$F$关于$\Omega$的灵敏度是有好处的,例如为了优化目的。换句话说,如果$R$增加,$F$会如何变化呢?现在我们看$F$关于$R$进行求导:
$$\frac{d}{dR}F=\int_{0}^{2\pi}\left( \frac{d}{dR}\int_{0}^{R}f(r,\theta;R)r\mathrm{d}r \right)\mathrm{d}\theta =\int_{0}^{2\pi}\int_{0}^{R}f^{'}(r,\theta;R)r\mathrm{d}r\mathrm{d}\theta + \int_{0}^{2\pi}f(R,\theta;R)R\mathrm{d}\theta$$
这里$f^{'}$是关于$R$求导。$f$对$R$的依赖性可以看做$f$对$\Omega$的依赖性。例如,$f(\cdot;R)\equiv f(\cdot;\Omega)$。使用笛卡尔坐标系(Cartesian coordinates)$\mathbf{x}$重新写上面公式有
\begin{gather}
\frac{d}{dR}F=\int_{\Omega}f^{'}(\mathbf{x};\Omega)\mathrm{d}\mathbf{x} + \int_{\partial \Omega}f(\mathbf{x};\Omega)\mathrm{d}S(\mathbf{x})
\end{gather}
这里$\mathrm{d}\mathbf{x}$是体积测度,$\mathrm{d}S(\mathbf{x})$是表面积测度。

\subsubsection{一般扰动}
令$\boldsymbol{v}$是$\partial \Omega$的单位外法向量。我们把增加的$R$看做速度场(velocity field)$\mathbf{V}$,在法方向上$\mathbf{V}$驱动$\Omega$上的点,取在$\Omega$上$\mathbf{V}=\boldsymbol{v}$。因此,方程$(2)$变成
\begin{gather}
\frac{d}{dR}F=\int_{\Omega}f^{'}(\mathbf{x};\Omega)\mathrm{d}\mathbf{x} + \int_{\partial \Omega}f(\mathbf{x};\Omega)\mathbf{V}(\mathbf{x})\cdot \boldsymbol{v}(\mathbf{x})\mathrm{d}S(\mathbf{x})
\end{gather}
我们把$\mathbf{V}$看成一个速度场,它会瞬间扰动(instantaneously perturbs domain)定义域$\Omega$。我们称$\mathbf{V}$是一个域扰动(domain perturbation)。现采用$f(\mathbf{x};\Omega)\equiv f(\Omega)$和$f^{'}(\mathbf{x};\Omega)\equiv f^{'}(\Omega;\mathbf{V})$,这里$f^{'}$是于域波动$\mathbf{V}$的$f$的形导数(shape derivative)。类似,我们定义$\delta F(\Omega;\mathbf{V})\equiv \frac{d}{dR}F$为$F$在方向$\mathbf{V}$上对于$\Omega$的形扰动(shape perturbation)。因此,我们有
\begin{gather}
\delta F(\Omega;\mathbf{V})=\int_{\Omega}f^{'}(\Omega;\mathbf{V})+\int_{\partial \Omega}f(\Omega)(\mathbf{V} \cdot \boldsymbol{v}),
\end{gather}
这是第五章中公式$(5.47)$,
$$\delta \varepsilon (\Omega;\mathbf{V})=\int_{\Omega}\dot{f}(\Omega;\mathbf{V})+\int_{\Omega}f(\Omega)(\nabla \cdot \mathbf{V})=\int_{\Omega}f^{'}(\Omega;\mathbf{V})+\int_{\partial \Omega}f(\Omega)(\mathbf{V} \cdot \boldsymbol{v})$$因此,我们导出公式$(5.47)$,其中对于$\Omega$是一个被速度场$\mathbf{V}$扰动的圆盘,使$\Omega$沿着法线方向上均匀扩大。这本书的主要目的是对于一般区域$\Omega$和一般扰动$\mathbf{V}$导出方程$(4)$,和其他类似的公式。

形微分的框架提供了建立平均曲率流(mean curvature flow)和Willmore流方程的工具,它们是流体力学和生物学等许多应用中的几何流。见第六章和第七章例子。

\subsubsection{Sequential Optimization of Shape}
怎么下山是显而易见的。只要你看见和感觉到地面,你就可以清楚地知道往哪个方向走才可以降低你的海拔。作为下一节动机,我们将其视为优化任务。换句话说,设$f=f(x,y)$是小山表面高度的函数,$(x,y)$是我们位置的对应。然后使用基本多变量微积分,找到一个下山方向相当于求$f$的梯度,并且沿着梯度相反的方向移动。所以我们不需要看整个函数,我们仅仅需要局部地计算梯度$\Delta f$,类似于感受下面的地面。

方程$(4)$的形扰动类似于梯度算子。它提供了局部的斜率,或者相当于某些参数敏感度的量。实际上,$(1.4)$是一个方向导数,类似与$\mathbf{V} \cdot \Delta f$,这里$\mathbf{V}$是一个方向。
\begin{tabular}{ |l|l|l|}   
\hline   
  &  标量函数 & 形状泛函 \\
\hline
量 & $f$ & F \\
\hline
参数 & $(x,y)$ & $\Omega$ \\
\hline
方向导数 & $\mathbf{V} \cdot \Delta f$ &  $\delta F(\Omega;\mathbf{V})$ \\
\hline
\end{tabular}
\\
上述表中类比并不相同,例如,只需要指定两个数$(x,y)$,而需要指定$\Omega$	无限个坐标对。$\mathbf{V}$是标量函数集合中的二维向量;对于形状函数,$\mathbf{V}$是一个完整的函数,需要在$\Omega$中的每一点上定义。这个“有限维数”是使用符号$\delta F(\Omega;\mathbf{V})$表示形状扰动。

因此,$\delta F(\Omega;\mathbf{V})$表示为了减少$F$我们应该如何改变$\Omega$,类似于为了减少$f$,如何改变$\Delta f(x,y)$和对应的坐标对$(x,y)$。形状优化$[3,23,48,51,54,59,75,93,106,107]$。下一节将介绍工程形状优化的一个经典示例。\\
\textbf{减少阻力}\\
虽然下面的例子超出了本书的范围,但它很好地展示了形状微分学的威力。考虑经过刚体的流(参见图$1.1$)。流向量速度场$\mathbf{u}$以无维形式遵从称为Navier-Stokes方程的偏微分方程$[8,79,101]$:
\begin{gather}
\begin{aligned}
(\mathbf{u}\cdot \Delta)\mathbf{u}-\Delta \cdot \sigma & =  0~~~in~\Omega,\\~\Delta \cdot \mathbf{u} & = 0~~~in~\Omega,\\~\mathbf{u}& =0~~~on~\Gamma _B,\\~\mathbf{u}& = e_x~~~on~\Gamma _O,
\end{aligned}
\end{gather}

\begin{gather}
\begin{aligned}
\sigma(\mathbf{u},p):& = -pI+\frac{1}{Re}D(\mathbf{u})\\D(\mathbf{u}):& = ~\Delta \mathbf{u}+(\Delta \mathbf{u})^T
\end{aligned}
\end{gather}
其中$0$为零向量,$e_x =(1,0)$为外边界$\Gamma _O$上的速度边界条件,$\sigma$为压力张量(stress tensor),$p$为压力。Reynolds数$Re$是与流的物理特性相关的无量纲参数。此时完全理解$(1.5)$和$(1.6)$并不重要。只要知道它们模拟了一个流如何在一个施加的流场中围绕一个固定的刚体运动就足够了。
\begin{figure}[H]
\centering
\includegraphics[scale=0.5]{./figures/11.png}
\caption{流体通过刚性$\Omega _B$ 的图。流体在$\Omega$中出现,由Navier-Stokes方程控制。$\Omega$的边界,记为$\partial \Omega$,划分为$\partial \Omega = \Gamma _B \cup \Gamma _O$,$\Gamma _B$ 为刚体的边界,$\Gamma _O$为远离刚体的外边界。$\Omega$的单位外法向量为$\nu$。}
\end{figure}

这里的目标是找到最佳形状的$\Omega _B$,以尽量减少对身体的阻力;这是形状优化中的一个经典问题$[42,69,82 -84]$。为此,我们需要指定一个形状函数代表阻力,即,
\begin{gather}
J_d(\Omega)=-e_x \cdot \int_{\Gamma _B}~\sigma(\mathbf{u},p)\nu,
\end{gather}
其中,我们使用$\Omega$来表示$\Omega _B$的形状,(这是因为$\Omega$和$\Omega _B$共享边界$\Gamma _B$。也可以证明$J_d$等于
\begin{gather}
J_d(\Omega)=\frac{1}{2Re}\int _{\Omega}\left| D(\mathbf{u}) \right|
\end{gather}
表示在$\Omega$域中能量的粘性耗散总量(单位时间)。注意,很明显,$J_d \geq 0$。利用形状扰动机制,$\delta J_d(\Omega;\mathbf{V})$表明$J_d$改变当我们在方向$\mathbf{V}$上扰动$\Omega$。因此,我们可以使用这些信息来微小的改变$\Omega$,慢慢变成一个形状,具有更好的(低级)阻力特性。

图$1.2$给出了一个数值计算来说明这一点。令$\Omega ^0$和$\Gamma _B$的初始猜测的形状的身体;这些在迭代$0$中显示。可以看到在身后面出现了两个大的涡流,这表明有大量的粘性耗散(大阻力)。优化过程,对于$\mathbf{V}$的许多不同的选择计算得到$\delta J_d(\Omega;\mathbf{V})$,至多选择一个驱动向下的$J_d$,最多。这个$\mathbf{V}$的选择用于在第一次迭代中将$\Gamma _B^0$变形为一个新的形状$\Gamma _B^1$,$\Gamma _B^0$在$\Gamma _B^1$和之间只有很小的差别。这个过程重复多次,结果如图$1.2$所示。注意旋涡(vorticses)是如何被更细的形状所消除的;显然,迭代$60$次时的对象比初始圆形具有更小的阻力。 
\begin{figure}[H]
\centering
\includegraphics[scale=0.5]{./figures/12.png}
\caption{通过图形优化阻力。从一个圆形的$\Gamma _B$(不是很符合空气动力学)开始,我们应用了一个最陡的下降优化方案来缓慢地进化到最小的$J_d$。蓝色曲线是速度场$\mathbf{u}$的流线,它满足$Re=200$的$(1.5)$
}
\end{figure}

因此,形状扰动使我们能够在无限维的形状中“爬下山”。这是一个自动生成复杂工程设计的强大工具。在图$1.2$中,这是在没有人决策干涉的优化模型。在创建仅有人类干涉的$1.5$的计算机模型,开发了一种生成形状序列的优化算法。
事实上,相同的优化机制可以用于不同的PDE系统,如弹性。描述形状优化的完整细节超出了本书的范围,但是在$6.3.1$节中有一个简短的讨论

\subsection{概念}
\subsubsection{向量}
所有向量变量都被认为是列向量,并以黑体符号表示。例如,$(q_1,...,q_n)$是$\mathbb{R}^n$中一个有$n$元素的行向量。如果$\mathbf{a}\in \mathbb{R}^n$是一个列向量,则我们可以写$\mathbf{a}=(a_1,a_2,a_3)^T$,此处$T$为转置算子。两个向量的点积可以写成$\mathbf{a} \cdot \mathbf{b}=\mathbf{a}^{T} \mathbf{b}$,其中两个向量为列向量。我们定义$|\mathbf{a}|$是向量$\mathbf{a}$的欧几里得范数。附录$A$中有关于向量,矩阵的基本概念和特性。
\subsubsection{梯度}
我们使用$t$是一个曲线参数,但是有时候类似于物理时间。对于曲面,用$s_1$,$s_2$表示参数变化量。符号$\Delta$是标准空间的梯度算子。所有向量导数算子(例如$\Delta$)都被认为行向量。如果$f=f(x,y,z)$是一个标量值函数,则$\Delta f$是一个$1\time 3$行向量。符号$\Delta _x$是关于变量$\mathbf{x}$的梯度,例如$\mathbf{x}=(x,y,z)$,则
$$
\\Delta _x=\left( \frac{\partial}{\partial x},\frac{\partial}{\partial y} ,\frac{\partial}{\partial z} \right).
$$
\subsubsection{积分}
我们通常使用$\Omega$作为$\mathbb{R}^3$中体积为正的定义域(或者$\mathbb{R}^2$)。我们定义$\Gamma$是$\mathbb{R}^3$的曲面,$\sum$是曲线。通常区域$D$s上特定映射表示$id_D$,例如$id_{\Omega}(\mathbf{x})=\mathbf{x}$,对$\Omega$中所有的$\mathbf{x}$,则
$\int _{\Omega} f(\mathbf{x})\mathrm{d}\mathbf{x}=\iiint_{\Omega} f(x_1,x_2,x_3)\,\mathrm{d}x_1\,\mathrm{d}x_2\,\mathrm{d}x_3.$
$\mathrm{d}\mathbf{x}$是体积积分,$\mathrm{d}S(\mathbf{x})$是面积积分,$\mathrm{d}a(\mathbf{x})$是曲线积分。此外,我们经常写积分如下
$$
\int _{\Omega} f(\mathbf{x})\mathrm{d}\mathbf{x} \equiv \int _{\Omega} f,~~~\int _{\Gamma} f(\mathbf{x})\mathrm{d}S(\mathbf{x}) \equiv \int _{\Gamma} f,~~~\int _{\sum} f(\mathbf{x})\mathrm{d}a(\mathbf{x}) \equiv \int _{\sum} f
$$
我们定义一个集合测度$|\cdot|$,例如
\begin{gather}
|\Omega|=\int _{\Omega}1,~~~|\Gamma|=\int _{\Gamma}1,~~~|\sum|=\int _{\sum}1
\end{gather}
因此,$|\Omega|$表示$\Omega$的体积,$|\Gamma|$表示$\Gamma$的表面积,$|\sum|$表示$\sum$的长度。




\input{2.1.tex}
\subsection{参数方法}
\subsubsection{什么是曲面}
曲面是空间中非常有规律(regular enough)的一组点。空间中点的随机分散不符合我们对曲面的直观概念。例如,它不够规律。另一方面,球面的边界确实符合我们对曲面的概念,也就是说,由于球面是“光滑的”,所以它具有足够的规律来作为曲面。
\begin{figure}[H]
\centering
\includegraphics[scale=0.5]{./figures/24.png}
\caption{参数表示的例子。参考域$U$是$x_1,x_2$平面上的方形显示。映射$X$将集合$U$对应于曲面$\Gamma$(曲面基于$x_3$坐标着色)。}
\end{figure}

直观地说,我们可以把创建一个表面想象成把一个平的橡胶板变形成一个弯曲的板。第$2.1.6$节中的变换$X$捕捉到了这个思想。因此,令$U \subset \mathbb{R}^2$是一个“平”域,令$X$:$U\Rightarrow \mathbb{R}^3$是个变形变换,例如,对于$U$中的每一个点$(s_1,s_2)^T$都对应于$\mathbb{R}^3$的点$x= (x_1,x_2, x_3)^T$使得
\begin{gather}
\mathbf{x}=X(s_1,s_2).
\end{gather}
令$\Gamma=X(U)$表示由“变形”$U$得到的曲面。我们称$(2.10)$是曲面$\Gamma$的参数表示,其中$s_1,s_2$称为表达参数。有时,我们把$U$作为一个参考域。有关$(2.10)$的示例,请参见图$2.4$。\\
\textbf{允许参数化}\\
如果我们要用$(2.10)$来定义曲面,那么我们必须对$X$进行假设,以保证$\Gamma=X(U)$是一个有效的曲面。最起码,X必须是连续的,以避免“撕裂”橡胶板。但是如果我们想对$\Gamma$做微积分,我们实际上需要更多。\\
\textbf{假设1.}我们对$X$做如下规律性假设。\\
$\bullet$ $(A1)$函数$X(x_1,x_2)$在$U$上是$C^{\infty}$,$\Gamma$中的每个点$\mathbf{x} = X(s_1,s_2)$对应于$U$中的一个点$(s_1, s_2)$,$X$是单射。
\begin{figure}[H]
\centering
\includegraphics[scale=0.5]{./figures/25.png}
\caption{这里的例子不满足假设$1$(参考域$U$没有显示)。$(a)$是一个圆锥体的参数表达,它在圆锥体的角上不能定义切平面。$(b)$是一个曲线,退化的曲面}
\end{figure}
$\bullet~(A2)$雅克比行列式(Jacobian)在$U$上秩为$2$,$J$ 的列向量是线性无关的。
\begin{gather}
J=\left[ \partial _{s_1}X,\partial _{s_2}X \right]=
\begin{bmatrix}
\partial _{s_1}X_1  & \partial _{s_2}X_1 \\
\partial _{s_1}X_2  & \partial _{s_2}X_2 \\
\partial _{s_1}X_3  & \partial _{s_2}X_3 
\end{bmatrix}
\end{gather}

我们说方程$(2.10)$的参数满足$(A1)$和$(A2)$是参数化的,用更现代的话说,immersion
\textbf{重要性}\\
假设$(A1)$对平面施加了一定的光滑性,即在$\Gamma$中的每一点上都有定义良好的切平面。切平面的精确定义见第$2.4.3$节;现在,我们只需要一个切平面的直观概念。例如,令$U=(-1,1)\times (-1,1)$并且考虑映射
\begin{gather}
X(s_1,s_2)=(s_1,s_2,\sqrt{s_1^2+s_2^2})^T~~~~~~~~(s_1,s_2)^T\in U.
\end{gather}
曲面$\Gamma = X(U)$是一个圆锥(见图$2.5(a)$)。很明显,$(2.12)$在$(0,0)^T$处不可导,即$(A1)$是无效的。因此,不存在唯一通过$(0,0,0)^T$且与表面$\Gamma$“相切”的平面。

需要假设$(A2)$来避免集合$\Gamma$(由$(2.10)$参数化)为$\mathbb{R}^3$中的曲线的可能性。通过线性代数,$A(2)$等价于$\partial _{s_1}X \times \partial _{s_2}X\neq 0$,这也等价于在$\mathbb{R}^3$中$\partial _{s_1}X$和$\partial _{s_2}X$是线性无关的向量。例如,令$U= (-1,1) \times (-1,1)$并且考虑映射
\begin{gather}
X(s_1,s_2)=(s_1+s_2,(s_1+s_2)^2,(s_1+s_2)^3)^T~~~~~~~~(s_1,s_2)^T\in U.
\end{gather}
“曲面”$\Gamma =X(U)$就是通过$X(t)=(t,t^2,t^3)^T$参数化后的曲线,其中t为参数(见图$2.5(b)$)。由$(2.13)$可知,$\partial _{s_1}X$和$\partial _{s_2}X$是线性相关的,即,在所有$U$上秩都是1,所以$(A2)$ 无效。因此,表面“退化”成曲线。

我们进一步描述$(A2)$。令$\mathbf{q}=(q_1,q_2)^T$为$U$中的一个点,定义$J^{q}=\left[ \partial _{s_1}X,\partial _{s_2}X \right]|_{s=q}$,(求出该点的雅可比矩阵)。注意$J_p$是$\mathbb{R}^{3\times 2}$中的一个常数矩阵。接下来,定义$T_q:\mathbb{R}^2 \Rightarrow \mathbb{R}^3$
\begin{gather}
T_q(\mathbf{p})=J_q \mathbf{p}~~~~~~~\Leftrightarrow~~~~~~~(T_q(\mathbf{p}))_i=\sum_{k=1}^{2}(J_q)_{ik}p_k~~~i=1,2,3,
\end{gather}
其中$\mathbf{p}= (p_1,p_2)$是$\mathbb{R}^2$中的任意一点。那么$(A2)$等价与映射$T_q$,是$U$中所有$\mathbf{q}$单射。集合$T_q(\mathbb{R})^2$是$J_q$中两个列向量产生;因此,它的维数是$2$。映射$T_q$与切平面有关,这将在$2.4.3$节中讨论。

\subsubsection{曲面参数化}
现在我们可以定义曲面的概念。

\textbf{定义1(参数曲面)}。令$U \subset \mathbb{R}^2$是一个开集并且考虑一个映射$X:\Rightarrow \mathbb{R}^3$。如果$X$在$U$中是可微的,我们称$(U,X)$是一个参数曲面。如果映射$T_q$对于所有$U$中的$\mathbf{q}$是单射,则称$X$是正则的(regular)。此外,如果有一个$U$中的$\mathbf{p}$不是单射,或未定义,则我们称$\mathbf{p}$时$X$的一个奇异点;否则,这是一个正则点。\\

注意,我们将这对$(U,X)$称为参数曲面,因为$\Gamma=X(U)$是构成曲面的点集,$U$和$X$描述了如何在$\Gamma$上“绘制”坐标曲线。进一步阐述可以见图$(2.4)$。参考域$U$只是组成一个正方形的一组点。然而,$U$上的网格线对应于$U$上的坐标系统,这些网格线通过$X$映射到$\Gamma$(见图$2.4$),这定义了$\Gamma$上的一种曲线坐标系。

如果我们在$U$上选择了一个不同的坐标系,那么网格线在$U$(和$\Gamma$)上看起来就会不一样。在$\Gamma$上会有一个不同的曲线坐标系(参见图$2.10$)。

因此,曲面可以以多种方式参数化就不足为奇了。实际上,给定$\Gamma$的一个参数化$(U,X)$
\begin{gather}
s_1=s_1(\tilde{s}_1,\tilde{s}_2),~~~s_2=s_2(\tilde{s}_1,\tilde{s}_2),~~~(\tilde{s}_1,\tilde{s}_2)^T \in ~\tilde{U}
\end{gather}
即,$s:\tilde{U} \rightarrow \mathbb{R}^2$和$U=s(\tilde{U})$。接下来,定义$\tilde{X}=X\circ s$,意味着
$$\tilde{X}(\tilde{s}_1,\tilde{s}_2)=X(s_1(\tilde{s}_1,\tilde{s}_2),s_2(\tilde{s}_1,\tilde{s}_2)).$$
这里$(\tilde{U},\tilde{X})$也是$\Gamma$的一个参数化。可以把$s$当成$\tilde{U}$到$U$的映射($s^{-1}$看成$U$到$\tilde{U}$的映射)
\begin{figure}[H]
\centering
\includegraphics[scale=0.5]{./figures/26.png}
\caption{所示图是不满假设$2$的例子。$(a)$是参考域。$(b)$是使用$(2.17)$中的映射$U$映射到一个环,注意这个环面覆盖了两次。$(c)$是使用$(2.18)$中的映射$U$映射到一个方形。网格线在$s_1$的附近挤压在一起}
\end{figure}

当然,为了有一个正则的参数化,我们必须有假设$1$必须满足$(\tilde{U},\tilde{X})$。这需要对$(2.15)$做以下假设。\\
\textbf{假设2.}\\
$\bullet~(A0*)$方程$(2.15)$被定义在区域$\tilde{U}$上且使得$U=s(\tilde{U})$。\\
$\bullet~(A1*)$方程$(2.15)$在$\tilde{U}$上是$C^{\infty}$,并且它是单射的。\\
$\bullet~(A2*)$雅克比矩阵
\begin{gather}
D=\left[ \partial _{\tilde{s}_1}s,\partial _{\tilde{s}_2}s \right]=
\begin{bmatrix}
\partial _{\tilde{s}_1}s_1  & \partial _{\tilde{s}_1}X_1 \\
\partial _{\tilde{s}_1}s_2  & \partial _{\tilde{s}_2}X_2 
\end{bmatrix}
\end{gather}
对于$\tilde{U}$上所有点$(\tilde{s}_1,\tilde{s}_2)$是非奇异的,即在$\tilde{U}$上$det(D)\neq 0$

我们说形如满足假设$2$的$(2.15)$的变换是一个允许的坐标变换(allowable)。

条件$(A1*)$和$(A2*)$彼此完全独立,实际上
\begin{gather}
s_1=e^{\tilde{s}_1}\cos (2\pi \tilde{s}_2),~~~~~s_2=e^{\tilde{s}_1}\sin (2\pi \tilde{s}_2)
\end{gather}
当$-1 \leq \tilde{s}_2 \leq 1$时是非单射的变换;然而,一个简单的计算给出$det(D) = (2\pi)^2e^{2\tilde{s}_2}$,它在$s_1,s_2$平面上从不为零(参见图$2.6$)。另一方面,变换
\begin{gather}
s_1=\tilde{s}^3 ,s_2=\tilde{s}
\end{gather}
处处都是单射,但是一个简单的计算得到$det(D)=3\tilde{s}_1^2=3\tilde{s}_1^{2/3}$,当$s=0$时,在整个$s_2$轴上它为零。(见图$2.6$)


\input{2.3.tex}
\input{2.4.tex}
\cite{tam19912d}
%\bibliography{../ref}
\end{document}
