% !Mode:: "TeX:UTF-8"
\documentclass[12pt,a4paper]{article}

%%%%%%%%------------------------------------------------------------------------
%%%% 日常所用宏包

%% 控制页边距
% 如果是beamer文档类, 则不用geometry
\makeatletter
\@ifclassloaded{beamer}{}{\usepackage[top=2.5cm, bottom=2.5cm, left=2.5cm, right=2.5cm]{geometry}}
\makeatother

%% 控制项目列表
\usepackage{enumerate}
\usepackage{framed}

%% 多栏显示
\usepackage{multicol}

%% 算法环境
\usepackage{algorithm}  
\usepackage{algorithmic} 
\usepackage{float} 

%% 网址引用
\usepackage{url}

%% 控制矩阵行距
\renewcommand\arraystretch{1.4}

%% 粗体
\usepackage{bm}


%% hyperref宏包,生成可定位点击的超链接,并且会生成pdf书签
\makeatletter
\@ifclassloaded{beamer}{
\usepackage{hyperref}
\usepackage{ragged2e} % 对齐
}{
\usepackage[%
    pdfstartview=FitH,%
    CJKbookmarks=true,%
    bookmarks=true,%
    bookmarksnumbered=true,%
    bookmarksopen=true,%
    colorlinks=true,%
    citecolor=blue,%
    linkcolor=blue,%
    anchorcolor=green,%
    urlcolor=blue%
]{hyperref}
}
\makeatother



\makeatletter % 如果是 beamer 不需要下面两个包
\@ifclassloaded{beamer}{
\mode<presentation>
{
} 
}{
%% 控制标题
\usepackage{titlesec}
%% 控制目录
\usepackage{titletoc}
}
\makeatother

%% 控制表格样式
\usepackage{booktabs}

%% 控制字体大小
\usepackage{type1cm}

%% 首行缩进,用\noindent取消某段缩进
\usepackage{indentfirst}

%% 支持彩色文本、底色、文本框等
\usepackage{color,xcolor}

%% AMS LaTeX宏包: http://zzg34b.w3.c361.com/package/maths.htm#amssymb
\usepackage{amsmath,amssymb}
%% 多个图形并排
\usepackage{subfloat}
%%%% 基本插图方法
%% 图形宏包
\usepackage{graphicx}
\newcommand{\red}[1]{\textcolor{red}{#1}}
\newcommand{\blue}[1]{\structure{#1}}
\newcommand{\brown}[1]{\textcolor{brown}{#1}}
\newcommand{\green}[1]{\textcolor{green}{#1}}


%%%% 基本插图方法结束

%%%% pgf/tikz绘图宏包设置
\usepackage{pgf,tikz}
\usetikzlibrary{shapes,automata,snakes,backgrounds,arrows}
\usetikzlibrary{mindmap}
%% 可以直接在latex文档中使用graphviz/dot语言,
%% 也可以用dot2tex工具将dot文件转换成tex文件再include进来
%% \usepackage[shell,pgf,outputdir={docgraphs/}]{dot2texi}
%%%% pgf/tikz设置结束


\makeatletter % 如果是 beamer 不需要下面两个包
\@ifclassloaded{beamer}{

}{
%%%% fancyhdr设置页眉页脚
%% 页眉页脚宏包
\usepackage{fancyhdr}
%% 页眉页脚风格
\pagestyle{plain}
}

%% 有时会出现\headheight too small的warning
\setlength{\headheight}{15pt}

%% 清空当前页眉页脚的默认设置
%\fancyhf{}
%%%% fancyhdr设置结束


\makeatletter % 对 beamer 要重新设置
\@ifclassloaded{beamer}{

}{
%%%% 设置listings宏包用来粘贴源代码
%% 方便粘贴源代码,部分代码高亮功能
\usepackage{listings}

%% 设置listings宏包的一些全局样式
%% 参考http://hi.baidu.com/shawpinlee/blog/item/9ec431cbae28e41cbe09e6e4.html
\lstset{
showstringspaces=false,              %% 设定是否显示代码之间的空格符号
numbers=left,                        %% 在左边显示行号
numberstyle=\tiny,                   %% 设定行号字体的大小
basicstyle=\scriptsize,                    %% 设定字体大小\tiny, \small, \Large等等
keywordstyle=\color{blue!70}, commentstyle=\color{red!50!green!50!blue!50},
                                     %% 关键字高亮
frame=shadowbox,                     %% 给代码加框
rulesepcolor=\color{red!20!green!20!blue!20},
escapechar=`,                        %% 中文逃逸字符,用于中英混排
xleftmargin=2em,xrightmargin=2em, aboveskip=1em,
breaklines,                          %% 这条命令可以让LaTeX自动将长的代码行换行排版
extendedchars=false                  %% 这一条命令可以解决代码跨页时,章节标题,页眉等汉字不显示的问题
}

\usepackage{minted}
\renewcommand{\listingscaption}{Python code} \newminted{python}{
    escapeinside=||,
    mathescape=true,
    numbersep=5pt,
    linenos=true,
    autogobble,
    framesep=3mm} 
}
\makeatother
%%%% listings宏包设置结束


%%%% 附录设置
\makeatletter % 对 beamer 要重新设置
\@ifclassloaded{beamer}{

}{
\usepackage[title,titletoc,header]{appendix}
}
\makeatother
%%%% 附录设置结束





%% 设定行距
\linespread{1}

\newcommand{\bfm}{\boldsymbol m}
\newcommand{\bfc}{\boldsymbol c}
\newcommand{\bfd}{\boldsymbol d}
\newcommand{\bfg}{\boldsymbol g}
\newcommand{\bff}{\boldsymbol f}
\newcommand{\bfx}{\boldsymbol x}
\newcommand{\bfu}{\boldsymbol u}
\newcommand{\bfn}{\boldsymbol n}
\newcommand{\bfv}{\boldsymbol v}
\newcommand{\bft}{\boldsymbol t}
\newcommand{\bfq}{\boldsymbol q}
\newcommand{\bfs}{\boldsymbol s}
\newcommand{\bfy}{\boldsymbol y}
\newcommand{\bfA}{\boldsymbol A}
\newcommand{\bfB}{\boldsymbol B}
\newcommand{\bfC}{\boldsymbol C}
\newcommand{\bfD}{\boldsymbol D}
\newcommand{\bfT}{\boldsymbol T}
\newcommand{\bfP}{\boldsymbol P}
\newcommand{\bfI}{\boldsymbol I}
\newcommand{\bfF}{\boldsymbol F}
\newcommand{\bfK}{\boldsymbol K}
\newcommand{\bfM}{\boldsymbol M}
\newcommand{\bfS}{\boldsymbol S}
\newcommand{\bfW}{\boldsymbol W}
\newcommand{\bfG}{\boldsymbol G}
\newcommand{\bfH}{\boldsymbol H}
\newcommand{\bfQ}{\boldsymbol Q}
\newcommand{\bfJ}{\boldsymbol J}
\newcommand{\balpha}{\bm \alpha}
\newcommand{\blambda}{\bm \lambda}
\newcommand{\bsigma}{\bm \sigma}
\newcommand{\bepsilon}{\bm \epsilon}
\newcommand{\bvarepsilon}{\bm \varepsilon}
\newcommand{\btau}{\bm \tau}
\newcommand{\rmd}{\,\mathrm d}
\newcommand{\cT}{\mathcal T}
\newcommand{\cF}{\mathcal F}
\newcommand{\cS}{\mathcal S}
\newcommand{\cP}{\mathcal P}
\newcommand{\cM}{\mathcal M}
\newcommand{\cA}{\mathcal A}
\newcommand{\cE}{\mathcal E}
\newcommand{\cB}{\mathcal B}
\newcommand{\cQ}{\mathcal Q}
\newcommand{\cN}{\mathcal N}
\newcommand{\cV}{\mathcal V}
\newcommand{\cW}{\mathcal W}
\newcommand{\bbS}{\mathbb S}
\newcommand{\bbR}{\mathbb R}
\newcommand{\od}{\text{div}}
\newcommand{\os}{\text{span}}
\newcommand{\ot}{\text{tr}}
\newcommand{\norm}[1]{||#1||}
\newcommand{\dof}{\text{dof}}

%%%% 个性设置结束
%%%%%%%%------------------------------------------------------------------------


%%%%%%%%------------------------------------------------------------------------
%%%% bibtex设置

%% 设定参考文献显示风格
% 下面是几种常见的样式
% * plain: 按字母的顺序排列,比较次序为作者、年度和标题
% * unsrt: 样式同plain,只是按照引用的先后排序
% * alpha: 用作者名首字母+年份后两位作标号,以字母顺序排序
% * abbrv: 类似plain,将月份全拼改为缩写,更显紧凑
% * apalike: 美国心理学学会期刊样式, 引用样式 [Tailper and Zang, 2006]

%\makeatletter
%\@ifclassloaded{beamer}{
%\bibliographystyle{apalike}
%}{
%\bibliographystyle{abbrv}
%}
%\makeatother


%%%% bibtex设置结束
%%%%%%%%------------------------------------------------------------------------

\input{../xecjk_preamble.tex}
\numberwithin{equation}{section}
\renewcommand {\thetable} {\thesection{}.\arabic{table}}
\renewcommand {\thefigure} {\thesection{}.\arabic{figure}}
\title{The Shapes of Things}
\author{龚欣}
\date{\chntoday}

\begin{document}
\maketitle

\section{第一章~~~~~~介绍}
\subsection{曲面上微分方程}
这本书提供了一个微分几何的介绍,这对含有几何片微分方程(PDE)的数学模型很有用,例如标准拉普拉斯方程的曲面或流形(manifold)。尤其是,这需要发展曲面梯度和曲面拉普拉斯算子。这些在标准欧几里得空间里(如$\mathbb{R}^n$)是普通的梯度$\bigtriangledown$和拉普拉斯算子$\bigtriangleup=\bigtriangledown \cdot \bigtriangledown$,但是在曲面或流体上定义就会变得特殊。\\

这种方法的优点是,它提供了几何量的替代公式,例如这些在微分几何中的公式,和(平均)曲率(curvature)比文章平时的表达式更清楚。

\subsection{关于形状的微分}
这本书中的微分几何方法有利于形状微分学的框架发展(framework of shape differential calculus),这是研究一个独立的“形状变量”的如何随其变化而变化的。

\subsubsection{一个简单例子}
接下来的例子需要freshman calculus的工具。令$f=f(r,\theta)$是定义在极坐标下半径为$R$的圆盘$\Omega$上,$F$是圆盘上的积分,如
\begin{gather}
F=\int_{\Omega} f=\int_{0}^{2\pi}\int_{0}^{R} f(r,\theta)r\mathrm{d}r\mathrm{d}\theta.
\end{gather}

显然,$F$依赖$R$,我们假设$f$总是依赖$R$,如$f=f(r,\theta;R)$。在物理例子中,$F$是液体通过横截面积为$\Omega$的管道的净流量(net flow rate)。在这种情况下,$f$是单位面积下的流速,可以是定义在$\Omega$上的偏微分方程的解。例如,在圆管中的Navier-Stokes流体。

知道$F$关于$\Omega$的灵敏度是有好处的,例如为了优化目的。换句话说,如果$R$增加,$F$会如何变化呢?现在我们看$F$关于$R$进行求导:
$$\frac{d}{dR}F=\int_{0}^{2\pi}\left( \frac{d}{dR}\int_{0}^{R}f(r,\theta;R)r\mathrm{d}r \right)\mathrm{d}\theta =\int_{0}^{2\pi}\int_{0}^{R}f^{'}(r,\theta;R)r\mathrm{d}r\mathrm{d}\theta + \int_{0}^{2\pi}f(R,\theta;R)R\mathrm{d}\theta$$
这里$f^{'}$是关于$R$求导。$f$对$R$的依赖性可以看做$f$对$\Omega$的依赖性。例如,$f(\cdot;\Omega)\equiv f(\cdot;\Omega)$。使用笛卡尔坐标系(Cartesian coordinates)$\mathbf{x}$重新写上面公式有
\begin{gather}
\frac{d}{dR}F=\int_{\Omega}f^{'}(\mathbf{x};\Omega)\mathrm{d}\mathbf{x} + \int_{\partial \Omega}f(\mathbf{x};\Omega)\mathrm{d}S(\mathbf{x})
\end{gather}
这里$\mathrm{d}\mathbf{x}$是体积测度,$\mathrm{d}S(\mathbf{x})$是表面积测度。

\subsubsection{一般扰动}
令$\boldsymbol{v}$是$\partial \Omega$的单位外法向量。我们把增加的$R$看做速度场(velocity field)$\mathbf{V}$,在法方向上$\mathbf{V}$驱动$\Omega$上的点,取在$\Omega$上$\mathbf{V}=\boldsymbol{v}$。因此,方程$(2)$变成
\begin{gather}
\frac{d}{dR}F=\int_{\Omega}f^{'}(\mathbf{x};\Omega)\mathrm{d}\mathbf{x} + \int_{\partial \Omega}f(\mathbf{x};\Omega)\mathbf{V}(\mathbf{x})\cdot \boldsymbol{v}(\mathbf{x})\mathrm{d}S(\mathbf{x})
\end{gather}
我们把$\mathbf{V}$看成一个速度场,它会瞬间扰动(instantaneously perturbs domain)定义域$\Omega$。我们称$\mathbf{V}$是一个域扰动(domain perturbation)。现采用$f(\mathbf{x};\Omega)\equiv f(\Omega)$和$f^{'}(\mathbf{x};\Omega)\equiv f^{'}(\Omega;\mathbf{V})$,这里$f^{'}$是于域波动$\mathbf{V}$的$f$的形导数(shape derivative)。类似,我们定义$\delta F(\Omega;\mathbf{V})\equiv \frac{d}{dR}F$为$F$在方向$\mathbf{V}$上对于$\Omega$的形扰动(shape perturbation)。因此,我们有
\begin{gather}
\delta F(\Omega;\mathbf{V})=\int_{\Omega}f^{'}(\Omega;\mathbf{V})+\int_{\partial \Omega}f(\Omega)(\mathbf{V} \cdot \boldsymbol{v}),
\end{gather}
这是第五章中公式$(5.47)$,
$$\delta \varepsilon (\Omega;\mathbf{V})=\int_{\Omega}\dot{f}(\Omega;\mathbf{V})+\int_{\Omega}f(\Omega)(\bigtriangledown \cdot \mathbf{V})=\int_{\Omega}f^{'}(\Omega;\mathbf{V})+\int_{\partial \Omega}f(\Omega)(\mathbf{V} \cdot \boldsymbol{v})$$因此,我们导出公式$(5.47)$,其中对于$\Omega$是一个被速度场$\mathbf{V}$扰动的圆盘,使$\Omega$沿着法线方向上均匀扩大。这本书的主要目的是对于一般区域$\Omega$和一般扰动$\mathbf{V}$导出方程$(4)$,和其他类似的公式。

形微分的框架提供了建立平均曲率流(mean curvature flow)和Willmore流方程的工具,它们是流体力学和生物学等许多应用中的几何流。见第六章和第七章例子。

\subsubsection{Sequential Optimization of Shape}
怎么下山是显而易见的。只要你看见和感觉到地面,你就可以清楚地知道往哪个方向走才可以降低你的海拔。作为下一节动机,我们将其视为优化任务。换句话说,设$f=f(x,y)$是小山表面高度的函数,$(x,y)$是我们位置的对应。然后使用基本多变量微积分,找到一个下山方向相当于求$f$的梯度,并且沿着梯度相反的方向移动。所以我们不需要看整个函数,我们仅仅需要局部地计算梯度$\triangledown f$,类似于感受下面的地面。

方程$(4)$的形扰动类似于梯度算子。它提供了局部的斜率,或者相当于某些参数敏感度的量。实际上,$(1.4)$是一个方向导数,类似与$\mathbf{V} \cdot \triangledown f$,这里$\mathbf{V}$是一个方向。
\begin{tabular}{ |l|l|l|}   
\hline   
  &  标量函数 & 形状泛函 \\
\hline
量 & $f$ & F \\
\hline
参数 & $(x,y)$ & $\Omega$ \\
\hline
方向导数 & $\mathbf{V} \cdot \triangledown f$ &  $\delta F(\Omega;\mathbf{V})$ \\
\hline
\end{tabular}
\\


上述表中类比并不相同,例如,只需要指定两个数$(x,y)$,而需要指定$\Omega$	无限个坐标对。

\subsection{概念}
\subsubsection{向量}
所有向量变量都被认为是列向量,并以黑体符号表示。例如,$(q_1,...,q_n)$是$\mathbb{R}^n$中一个有$n$元素的行向量。如果$\mathbf{a}\in \mathbb{R}^n$是一个列向量,则我们可以写$\mathbf{a}=(a_1,a_2,a_3)^T$,此处$T$为转置算子。两个向量的点积可以写成$\mathbf{a} \cdot \mathbf{b}=\mathbf{a}^{T} \mathbf{b}$,其中两个向量为列向量。我们定义$|\mathbf{a}|$是向量$\mathbf{a}$的欧几里得范数。附录$A$中有关于向量,矩阵的基本概念和特性。
\subsubsection{梯度}
我们使用$t$是一个曲线参数,但是有时候类似于物理时间。对于曲面,用$s_1$,$s_2$表示参数变化量。符号$\triangledown$是标准空间的梯度算子。所有向量导数算子(例如$\triangledown$)都被认为行向量。如果$f=f(x,y,z)$是一个标量值函数,则$\triangledown f$是一个$1\time 3$行向量。符号$\triangledown _x$是关于变量$\mathbf{x}$的梯度,例如$\mathbf{x}=(x,y,z)$,则
$$
\triangledown _x=\left( \frac{\partial}{\partial x},\frac{\partial}{\partial y} ,\frac{\partial}{\partial z} \right).
$$
\subsubsection{积分}
我们通常使用$\Omega$作为$\mathbb{R}^3$中体积为正的定义域(或者$\mathbb{R}^2$)。我们定义$\Gamma$是$\mathbb{R}^3$的曲面,$\sum$是曲线。通常区域$D$s上特定映射表示$id_D$,例如$id_{\Omega}(\mathbf{x})=\mathbf{x}$,对$\Omega$中所有的$\mathbf{x}$,则
$\int _{\Omega} f(\mathbf{x})\mathrm{d}\mathbf{x}=\iiint_{\Omega} f(x_1,x_2,x_3)\,
\mathrm{d}x_1\,\mathrm{d}x_2\,\mathrm{d}x_3.$
$\mathrm{d}\mathbf{x}$是体积积分,$\mathrm{d}S(\mathbf{x})$是面积积分,$\mathrm{d}a(\mathbf{x})$是曲线积分。此外,我们经常写积分如下
$$
\int _{\Omega} f(\mathbf{x})\mathrm{d}\mathbf{x} \equiv \int _{\Omega} f,~~~\int _{\Gamma} f(\mathbf{x})\mathrm{d}S\mathbf{x} \equiv \int _{\Gamma} f,~~~\int _{\sum} f(\mathbf{x})\mathrm{d}a\mathbf{x} \equiv \int _{\sum} f
$$
我们定义一个集合测度$|\cdot|$,例如
\begin{gather}
|\Omega|=\int _{\Omega}1,~~~|\Gamma|=\int _{\Gamma}1,~~~|\sum|=\int _{\sum}1
\end{gather}
因此,$|\Omega|$表示$\Omega$的体积,$|\Gamma|$表示$\Gamma$的表面积,$|\sum|$表示$\sum$的长度。

\section{第二章~~~~~~~曲面和微分几何}

微分几何是对曲面(流形)形状的详细研究,包括局部和全局性质。$\mathbb{R}^3$(三维)中的平面是一个非常简单的曲面,不需要特殊的工具来描述它。另一方面,一个“任意”形状的表面,如汽车的引擎盖,有许多明显的几何特征。(高度弯曲区域、几乎平的区域等)。定量和定性地描述这些特征需要微分几何的工具。此外,几何细节在许多物理和生物过程中都很重要,例如表面张力$[20,21]$和生物膜$[9,55,90,114]$。

微分几何的框架首先通过定义一个局部映射(如,曲面参数)。然后,在曲面上建立了一个类似于标准“欧几里德微积分”的微积分框架。也可以采用其他方法,例如使用由级别集(level sets)和距离函数定义的隐式曲面。尽管是任意的,但是参数化在各种设置中都非常有用,所以我们将主要使用这些。我们强调曲面的几何形状不依赖于特定的参数化;在$2.3$节中引入正则曲面的概念来处理这个问题(参见命题$1$)。
在这一章以及本书的其余部分,我们主要关注三维空间中的二维曲面。我们首先回顾一些基础知识,以便使本文尽可能完整。

\subsection{预备}
下面几节将快速回顾本书的基本概念。然而,为了阅读本书,理解集合、映射等的所有细节并不重要。但是,如果这里讨论的观点与您完全不同,那么我们鼓励您参考一本好的教科书,如[61,62,64]
\subsubsection{欧几里得空间}
$\mathbb{R}^n$表示$n$维欧氏空间。在整个文本中,我们主要取$n=3$,但有时我们可能会专门化成$n=2$。我们假设读者熟悉笛卡尔坐标系、向量符号和向量算法、向量运算点积、叉乘、两个向量之间的夹角等。$\mathbb{R}^3$中的一般向量$\mathbf{x}$通常有由$\mathbf{x}= (x_1, x_2, x_3)$表示。

接下来,假设我们有一个给定的坐标系。任何点$P$在$\mathbb{R}^n$中都有一个唯一的位置向量,即$\mathbf{x}_P\in \mathbb{R}^n$,它从原点到$P$。点$P$的坐标就是向量$\mathbf{x}_P$的分量。因此,有时用向量$\mathbf{x}_P$来表示点$P$是很方便的,这意味着我们将通过位置矢量来表示该点。这种情况下,我们将去掉下标,只引用点$\mathbf{x}$。当没有歧义的时,我们将充分使用这种符号。否则,我们将强调点和位置向量之间的区别。
\subsubsection{开闭集合,边界,领域}
一般来说,集合是不同对象的集合。例如,$\lbrace  X,Y \rbrace$是由不同的对象$X$和$Y$组成的集合;我们在定义一个集合时使用大括号$\lbrace  , \rbrace$。我们经常引入另一个符号,如$Q=\lbrace X , Y \rbrace$,为了方便引用集合。设$S$和$U$为集合,它们之间存在关系有交集,补集,并集等。

有时候一个集合是通过一个条件定义的。例如,$ \lbrace x\in G$:$x$要满足条件$\rbrace$,例如集合$\lbrace 1,2,3\rbrace$也可以由$a \in \mathbb{Z}:a>0~and~a<4$定义,其中$\mathbb{Z}$是整数集。空集表示为$\emptyset $,是唯一一个没有元素的集合:$\lbrace  ~ \rbrace$。

给定$\mathbb{R}^n$中一个点$\mathbf{x}$,和一个正数$r$,设$B_r(\mathbf{x})$是$\mathbb{R}^n$中所有到点$\mathbf{x}$的距离严格小于$r$的集合,表达如下
\begin{gather}
B_r(\mathbf{x})=\left\{ \mathbf{y} \in \mathbb{R}^n:\left| \mathbf{x} - \mathbf{y} \right| <r \right\}
\end{gather}
换句话说,$B_r(\mathbf{x})$是一个以$\mathbf{x}$为中心半径为$r$的实心球($n$维)的内部。接下来,我们定义$B_r(\mathbf{x})$的边界为
\begin{gather}
\partial B_r(\mathbf{x})=\left\{ \mathbf{y} \in \mathbb{R}^n:\left| \mathbf{x} - \mathbf{y} \right|=r  \right\}
\end{gather}
即$\partial B_r(\mathbf{x})$是以$\mathbf{x}$为中心半径为$r$的球面。
从上面我们可以看到$B_r(\mathbf{x}) \cap \partial B_r(\mathbf{x})$是空集,$B_r(\mathbf{x})$不包含$\partial B_r(\mathbf{x})$的任何部分。换句话说,$B_r(\mathbf{x})$不包含其边界的任何部分。更正式的写法是$B_r(\mathbf{x}) \cap \partial B_r(\mathbf{x})=\emptyset$。

我们使用术语$open$表示一个集合不包含其边界的任何部分。更准确地说,一个子集$U$在$\mathbb{R}^n$是开的,如果$U$中每个点都有$B_r(\mathbf{x})$,其中$r>0$,且包含在$U$中。换句话说,给开集$U$中一点$\mathbf{x}$,我们可以找到一个包含$\mathbf{x}$的球完全包含在$U$中。事实上,我们经常把$B_r(\mathbf{x})$看成一个开球。另一个开放集的例子是$(0,1)\subset R$ 。 即$O$和$1$之间的组数,但不包括$0$和$1$。

一个$S\in \mathbb{R}^n$的集合的边界是$\mathbb{R}^n$中的点,使得包含此点的开集即包含有$S$中的点又包含有不是$S$中的点。换句话说,$S$中边界点$\mathbf{x}$,对于任意$r$,都满足$B_r(\mathbf{x}) \cap S=\emptyset$和$B_r(\mathbf{x}) \cap (\mathbb{R}^n \ S)=\emptyset$。我们用$\partial S$表示$S$的边界,参见图$2.1$的图形说明。

我们用闭的说明集合包含所有边界,沿着这些线,用$\bar{S}$表示$S$的闭包,等于$\bar{S}=S \cup \partial S$。因此我们有一个以$\mathbf{x}$为圆心,半径为$r$闭合球$\overline{B_r(\mathbf{x})}$表示为
\begin{gather}
\overline{B_r(\mathbf{x})}=\left\{ \mathbf{y} \in \mathbb{R}^n:|\mathbf{x}- \mathbf{y}| \leq r \right\}
\end{gather}
另一个闭集例子$[0,1]\subset \mathbb{R}$,这个闭集即包含$0$到$1$之间的数,又包含$0$和$1$。

\textbf{备注2.}在这本书中,我们经常围绕一个点来定义一个领域,如,点$\mathbf{x}$的邻域,是任何包含$\mathbf{x}$的开集$U \subset \mathbb{R}^n$。

\subsubsection{紧集}
如果一个集合包含在$\mathbb{R}^n$中一个足够大但半径有限的开球中,那么它就是有界的。此外,如果$\mathbb{R}^n$中的集合是闭的和有界的,那么它就是紧集。紧性的概念实际上比这更普遍[63,64]。但是对于我们的目的,前面的定义是充分的。

如果$\bar{S} \subset W$和$\bar{S}$是紧的,我们说一个(非空)开集$S$紧包含在另一个开集$W$中,记为$S \subset \subset W$。也就是说,$S$的边界不能与$W$的边界接触,即,在$\partial S$和$\partial W$之间有“一点空间”。有了这个,我们现在可以证明紧支撑函数的概念。假设$f$是定义在$S$上的函数,$f$的支撑被定义为$S$中$f$的函数值非零的点的集合。
$$supp (f)= \left\{ \mathbf{x} \in S :f(\mathbf{x}) \neq 0 \right\}$$
此外,如果$\overline{supp (f)} \subset \subset S$,我们就说$f$是$S$中紧支撑。
\begin{figure}[H]
\centering
\includegraphics[scale=0.5]{./figures/22.png}
\caption{$\mathbb{R}^n$中的集合$S$通过映射$\Phi$映射到$S^{'}$上。对于$i=1,...,5$这些点对应于$\mathbf{x}_i^{'}=\Phi(\mathbf{x})$。我们可以把$\Phi$的作用解释为集合$S$变形为$S^{'}$,也就是说,$S$通过变换$\Phi$变成$S^{'}$的形状。}
\end{figure}

\textbf{备注3.}紧支持是忽略边界的有用影响。对于本书中的一些证明,我们需要这个概念来保持“一个泛函的作用”远离一个集合的边界,或者在一个感兴趣的区域内局部化一个函数。一个原因是为了避免在定义的边界上函数进行微分会产生潜在的差异。或者,更常见的是,我们希望忽略一个依赖于函数在边界点的值的量。例如,如果$f$在$S$上有紧支撑,则$\int _{\partial S}f=0$。

\subsubsection{映射:基本定义}
$S$和$S^{'}$是两组集合。如果有一个“规则”(函数)$\Phi$,使$S$中的每一点都有$S^{'}$中的点$\mathbf{x}^{'}$与之对应,则我们说这是一个集合$S$到$S^{'}$的映射或变换。我们使用符号$\Phi$:$S\rightarrow S^{'}$表达前面语句的简称。
有了这个,我们可以写成$\mathbf{x}^{'}=\Phi (\mathbf{x})$我们称$\mathbf{x}^{'}$为$\mathbf{x}$的像点,$\mathbf{x}$称为$\mathbf{x}^{'}$的逆像点。

\textbf{备注4.}一般来说,如果$S\subset \mathbb{R}^m$和$S^{'}\subset \mathbb{R}^n$,则
\begin{gather}
\Phi =(\Phi _1,\Phi _2,...,\Phi_n)^T
\end{gather}
其中每个$\Phi _i$是一个$m$个参量的函数:$\Phi _i = \Phi _i(x_1,x_2,...,x_m)$。
\begin{figure}[H]
\centering
\includegraphics[scale=0.5]{./figures/21.png}
\caption{}
\end{figure}
$S$中所有点的像点集合称为$S$的像,记为$\Phi (S)$。如果$S^{'}$中每一点都是$S$中的像点,则映射$\Phi$将$S$映到$S^{'}$上,即$S^{'}=\Phi (S)$
。在这种情况下,我们称$\Phi (S)$为满射。参见图$2.2$是$\mathbb{R}^2$中的点集映射的一个例子(参见图$2.3$是$\mathbb{R}^3$一组映射的例子)。

如$S$中任意一对不同点的像点也是$S^{'}$中的不同点,那么我们称$\Phi (S)$是单射(也叫一对一映射)。$\Phi$即是满射又是单射(称为双射),则存在$\Phi$的逆映射,记$\Phi ^{-1}$,将$S^{'}$中的点映射到$S$上。即如果$\mathbf{x},\mathbf{x}^{'}$满足$\mathbf{x}^{'}= \Phi (\mathbf{x} )$,则$\mathbf{x}=\Phi ^{-1} (\mathbf{x}^{'})$,$\Phi ^{-1} :S^{'}\rightarrow S$

$S$到$S^{'}$的映射$\Phi$在$S$中的$\mathbf{x}$点处是连续的,如果对于任意包含$\mathbf{x}^{'}= \Phi (\mathbf{x})$的一个领域$N^{'}$,存在$\mathbf{x}$的一个领域$N$,使得$\Phi (N)\subset N^{'}$。如果它在$S$的每一点都是连续的,我们称该映射是连续的。

双射$\Phi$是连续映射,并且其逆$\Phi^{-1}$也是连续的,则称为拓扑映射或同胚。点集可以拓扑地互映射到其他点集称他们为同胚的。同胚集合具有相同的“拓扑”,即,它们的连通性是一样的;它们有相同类型的“洞”。在$2.3.1$节中对此有进一步的讨论,图$2.7$显示了当映射不是同胚时可以发生什么。

如果任意两点$\mathbf{a}$和$\mathbf{b}$的距离等于$\Phi(\mathbf{a})$和$\Phi(\mathbf{b})$的距离,映射$\Phi$就称为刚性运动(rigid motion)。

\subsubsection{正交变换}
设$b = (b_1, b_2, b_2)\in~\mathbb{R}^3$,$A\in~\mathbb{R}^{3\times 3}$,即一个$3\times 3$矩阵$A=[a_{ij}]_{i,j=1}^3$,$a_{i,j}$是矩阵$A$的元素。定义以下(仿射affine)线性映射$\Phi$(转换):
\begin{gather}
\tilde{\mathbf{x}}=\Phi (\mathbf{x})=A\mathbf{x}+b~~~~~~~~\Leftrightarrow~~~~~~~~\tilde{x_i}=\Phi (\mathbf{x})_i=\left( \sum _{k=1}^{3}a_{ik}x_{k} \right)+b_i,
\end{gather}
这里$(\Phi(\mathbf{x}))_i\equiv \Phi _i(x_1,x_2,x_3)$。如果$A$满足下面性质
\begin{gather}
A^{-1}=A^{T},~~~~~~~~~det(A)=1,
\end{gather}
其中$det(A)$为$A$的行列式,则$\Phi$表示刚体运动。基本上,$\Phi$由一个旋转(rotation)($A$),后跟一个翻译(translation)($b$)。一个刚性运动可以被用来从一个笛卡儿坐标系统转换到另一个坐标系。

如果$b=0$并且$(2.6)$仍然成立,则$\Phi(\mathbf{x})=A\mathbf{x}$是一个线性映射称为正交变化(direct orthbogonal transformation)。这不过是以原点为中心的坐标系的旋转。如果$(2.6)$被替换为
\begin{gather}
A^{-1}=A^{T},~~~~~~~~~det(A)=-1,
\end{gather}
则$\Phi(\mathbf{x})=A\mathbf{x}$称为反正交变换,$(2.6)$和$(2.7)$都是正交矩阵的。\\

\textbf{备注4(转换的解释).}。我们可以用两种不同的方法来解释$(2.5)$。考虑$\\mathbb{R}^3$坐标为$\mathbf{x}$的点$P$。\\

$\bullet$Alias.将$(2.5)$作为坐标的变换,$\mathbf{x}$和$\tilde{\mathbf{x}}$是相对于不同坐标系同一个点的坐标。换句话说,该点由不同的“名称”。\\

$\bullet$Alibi.将$(2.5)$看成集合的映射,$\mathbf{x}$和$\tilde{\mathbf{x}}$是同一坐标系下不同点的坐标。换句话说,点$\tilde{\mathbf{x}}$是点$\mathbf{x}$映射之前的点。

质点(material)的概念与alibi的观点直接相关,我们可以想象一个物质的“粒子”(即质点),最初在$\mathbf{x}$,然后因为某种物理过程而转移到$\tilde{\mathbf{x}}$点。转换$(2.5)$简单地表示物理过程的运动结果。在变形连续介质力学中是标准概念,尤其是非线性弹性力学中。图$2.3$是$\mathbb{R}^3$中点集的一个刚体运动。

\begin{figure}[H]
\centering
\includegraphics[scale=0.5]{./figures/23.png}
\caption{$(a)$是$\mathbb{R}^3$中点集$S$。$(b)$是$S^{'}=\Phi (S)$的旋转集合,其中$\Phi$是$(2.5)$和$(2.6)$中定义的。$(c)$是$S^{'}=\Phi (S)$的变形集合,其中$\Phi$是$(2.9)$中定义的。}
\end{figure}




\input{test.tex}
\cite{tam19912d}
%\bibliography{../ref}
\end{document}
