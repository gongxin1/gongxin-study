% !Mode:: "TeX:UTF-8"
\documentclass[12pt,a4paper]{article}

%%%%%%%%------------------------------------------------------------------------
%%%% 日常所用宏包

%% 控制页边距
% 如果是beamer文档类, 则不用geometry
\makeatletter
\@ifclassloaded{beamer}{}{\usepackage[top=2.5cm, bottom=2.5cm, left=2.5cm, right=2.5cm]{geometry}}
\makeatother

%% 控制项目列表
\usepackage{enumerate}
\usepackage{framed}

%% 多栏显示
\usepackage{multicol}

%% 算法环境
\usepackage{algorithm}  
\usepackage{algorithmic} 
\usepackage{float} 

%% 网址引用
\usepackage{url}

%% 控制矩阵行距
\renewcommand\arraystretch{1.4}

%% 粗体
\usepackage{bm}


%% hyperref宏包,生成可定位点击的超链接,并且会生成pdf书签
\makeatletter
\@ifclassloaded{beamer}{
\usepackage{hyperref}
\usepackage{ragged2e} % 对齐
}{
\usepackage[%
    pdfstartview=FitH,%
    CJKbookmarks=true,%
    bookmarks=true,%
    bookmarksnumbered=true,%
    bookmarksopen=true,%
    colorlinks=true,%
    citecolor=blue,%
    linkcolor=blue,%
    anchorcolor=green,%
    urlcolor=blue%
]{hyperref}
}
\makeatother



\makeatletter % 如果是 beamer 不需要下面两个包
\@ifclassloaded{beamer}{
\mode<presentation>
{
} 
}{
%% 控制标题
\usepackage{titlesec}
%% 控制目录
\usepackage{titletoc}
}
\makeatother

%% 控制表格样式
\usepackage{booktabs}

%% 控制字体大小
\usepackage{type1cm}

%% 首行缩进,用\noindent取消某段缩进
\usepackage{indentfirst}

%% 支持彩色文本、底色、文本框等
\usepackage{color,xcolor}

%% AMS LaTeX宏包: http://zzg34b.w3.c361.com/package/maths.htm#amssymb
\usepackage{amsmath,amssymb}
%% 多个图形并排
\usepackage{subfloat}
%%%% 基本插图方法
%% 图形宏包
\usepackage{graphicx}
\newcommand{\red}[1]{\textcolor{red}{#1}}
\newcommand{\blue}[1]{\structure{#1}}
\newcommand{\brown}[1]{\textcolor{brown}{#1}}
\newcommand{\green}[1]{\textcolor{green}{#1}}


%%%% 基本插图方法结束

%%%% pgf/tikz绘图宏包设置
\usepackage{pgf,tikz}
\usetikzlibrary{shapes,automata,snakes,backgrounds,arrows}
\usetikzlibrary{mindmap}
%% 可以直接在latex文档中使用graphviz/dot语言,
%% 也可以用dot2tex工具将dot文件转换成tex文件再include进来
%% \usepackage[shell,pgf,outputdir={docgraphs/}]{dot2texi}
%%%% pgf/tikz设置结束


\makeatletter % 如果是 beamer 不需要下面两个包
\@ifclassloaded{beamer}{

}{
%%%% fancyhdr设置页眉页脚
%% 页眉页脚宏包
\usepackage{fancyhdr}
%% 页眉页脚风格
\pagestyle{plain}
}

%% 有时会出现\headheight too small的warning
\setlength{\headheight}{15pt}

%% 清空当前页眉页脚的默认设置
%\fancyhf{}
%%%% fancyhdr设置结束


\makeatletter % 对 beamer 要重新设置
\@ifclassloaded{beamer}{

}{
%%%% 设置listings宏包用来粘贴源代码
%% 方便粘贴源代码,部分代码高亮功能
\usepackage{listings}

%% 设置listings宏包的一些全局样式
%% 参考http://hi.baidu.com/shawpinlee/blog/item/9ec431cbae28e41cbe09e6e4.html
\lstset{
showstringspaces=false,              %% 设定是否显示代码之间的空格符号
numbers=left,                        %% 在左边显示行号
numberstyle=\tiny,                   %% 设定行号字体的大小
basicstyle=\scriptsize,                    %% 设定字体大小\tiny, \small, \Large等等
keywordstyle=\color{blue!70}, commentstyle=\color{red!50!green!50!blue!50},
                                     %% 关键字高亮
frame=shadowbox,                     %% 给代码加框
rulesepcolor=\color{red!20!green!20!blue!20},
escapechar=`,                        %% 中文逃逸字符,用于中英混排
xleftmargin=2em,xrightmargin=2em, aboveskip=1em,
breaklines,                          %% 这条命令可以让LaTeX自动将长的代码行换行排版
extendedchars=false                  %% 这一条命令可以解决代码跨页时,章节标题,页眉等汉字不显示的问题
}

\usepackage{minted}
\renewcommand{\listingscaption}{Python code} \newminted{python}{
    escapeinside=||,
    mathescape=true,
    numbersep=5pt,
    linenos=true,
    autogobble,
    framesep=3mm} 
}
\makeatother
%%%% listings宏包设置结束


%%%% 附录设置
\makeatletter % 对 beamer 要重新设置
\@ifclassloaded{beamer}{

}{
\usepackage[title,titletoc,header]{appendix}
}
\makeatother
%%%% 附录设置结束





%% 设定行距
\linespread{1}

\newcommand{\bfm}{\boldsymbol m}
\newcommand{\bfc}{\boldsymbol c}
\newcommand{\bfd}{\boldsymbol d}
\newcommand{\bfg}{\boldsymbol g}
\newcommand{\bff}{\boldsymbol f}
\newcommand{\bfx}{\boldsymbol x}
\newcommand{\bfu}{\boldsymbol u}
\newcommand{\bfn}{\boldsymbol n}
\newcommand{\bfv}{\boldsymbol v}
\newcommand{\bft}{\boldsymbol t}
\newcommand{\bfq}{\boldsymbol q}
\newcommand{\bfs}{\boldsymbol s}
\newcommand{\bfy}{\boldsymbol y}
\newcommand{\bfA}{\boldsymbol A}
\newcommand{\bfB}{\boldsymbol B}
\newcommand{\bfC}{\boldsymbol C}
\newcommand{\bfD}{\boldsymbol D}
\newcommand{\bfT}{\boldsymbol T}
\newcommand{\bfP}{\boldsymbol P}
\newcommand{\bfI}{\boldsymbol I}
\newcommand{\bfF}{\boldsymbol F}
\newcommand{\bfK}{\boldsymbol K}
\newcommand{\bfM}{\boldsymbol M}
\newcommand{\bfS}{\boldsymbol S}
\newcommand{\bfW}{\boldsymbol W}
\newcommand{\bfG}{\boldsymbol G}
\newcommand{\bfH}{\boldsymbol H}
\newcommand{\bfQ}{\boldsymbol Q}
\newcommand{\bfJ}{\boldsymbol J}
\newcommand{\balpha}{\bm \alpha}
\newcommand{\blambda}{\bm \lambda}
\newcommand{\bsigma}{\bm \sigma}
\newcommand{\bepsilon}{\bm \epsilon}
\newcommand{\bvarepsilon}{\bm \varepsilon}
\newcommand{\btau}{\bm \tau}
\newcommand{\rmd}{\,\mathrm d}
\newcommand{\cT}{\mathcal T}
\newcommand{\cF}{\mathcal F}
\newcommand{\cS}{\mathcal S}
\newcommand{\cP}{\mathcal P}
\newcommand{\cM}{\mathcal M}
\newcommand{\cA}{\mathcal A}
\newcommand{\cE}{\mathcal E}
\newcommand{\cB}{\mathcal B}
\newcommand{\cQ}{\mathcal Q}
\newcommand{\cN}{\mathcal N}
\newcommand{\cV}{\mathcal V}
\newcommand{\cW}{\mathcal W}
\newcommand{\bbS}{\mathbb S}
\newcommand{\bbR}{\mathbb R}
\newcommand{\od}{\text{div}}
\newcommand{\os}{\text{span}}
\newcommand{\ot}{\text{tr}}
\newcommand{\norm}[1]{||#1||}
\newcommand{\dof}{\text{dof}}

%%%% 个性设置结束
%%%%%%%%------------------------------------------------------------------------


%%%%%%%%------------------------------------------------------------------------
%%%% bibtex设置

%% 设定参考文献显示风格
% 下面是几种常见的样式
% * plain: 按字母的顺序排列,比较次序为作者、年度和标题
% * unsrt: 样式同plain,只是按照引用的先后排序
% * alpha: 用作者名首字母+年份后两位作标号,以字母顺序排序
% * abbrv: 类似plain,将月份全拼改为缩写,更显紧凑
% * apalike: 美国心理学学会期刊样式, 引用样式 [Tailper and Zang, 2006]

%\makeatletter
%\@ifclassloaded{beamer}{
%\bibliographystyle{apalike}
%}{
%\bibliographystyle{abbrv}
%}
%\makeatother


%%%% bibtex设置结束
%%%%%%%%------------------------------------------------------------------------

\input{../xecjk_preamble.tex}

\title{The Shapes of Things}
\author{龚欣}
\date{\chntoday}

\begin{document}
\maketitle
\section{第一章}
\subsection{曲面上微分方程}
这本书提供了一个微分几何的介绍,这对含有几何片微分方程(PDE)的数学模型很有用,例如标准拉普拉斯方程的曲面或流形(manifold)。尤其是,这需要发展曲面梯度和曲面拉普拉斯算子。这些在标准欧几里得空间里(如$\mathbb{R}^n$)是普通的梯度$\bigtriangledown$和拉普拉斯算子$\bigtriangleup=\bigtriangledown \cdot \bigtriangledown$,但是在曲面或流体上定义就会变得特殊。\\

这种方法的优点是,它提供了几何量的替代公式,例如这些在微分几何中的公式,和(平均)曲率(curvature)比文章平时的表达式更清楚。

\subsection{关于形状的微分}
这本书中的微分几何方法有利于形状微分学的框架发展(framework of shape differential calculus),这是研究一个独立的“形状变量”的如何随其变化而变化的。

\subsubsection{一个简单例子}
接下来的例子需要freshman calculus的工具。令$f=f(r,\theta)$是定义在极坐标下半径为$R$的圆盘$\Omega$上,$F$是圆盘上的积分,如
\begin{equation}
\begin{aligned}
F=\int_{\Omega} f=\int_{0}^{2\pi}\int_{0}^{R} f(r,\theta)r\mathrm{d}r\mathrm{d}\theta.
\end{aligned}
\end{equation}
显然,$F$依赖$R$,我们假设$f$总是依赖$R$,如$f=f(r,\theta;R)$。在物理例子中,$F$是液体通过横截面积为$\Omega$的管道的净流量(net flow rate)。在这种情况下,$f$是单位面积下的流速,可以是定义在$\Omega$上的偏微分方程的解。例如,在圆管中的Navier-Stokes流体。

知道$F$关于$\Omega$的灵敏度是有好处的,例如为了优化目的。换句话说,如果$R$增加,$F$会如何变化呢?现在我们看$F$关于$R$进行求导:
$$\frac{d}{dR}F=\int_{0}^{2\pi}\left( \frac{d}{dR}\int_{0}^{R}f(r,\theta;R)r\mathrm{d}r \right)\mathrm{d}\theta =\int_{0}^{2\pi}\int_{0}^{R}f^{'}(r,\theta;R)r\mathrm{d}r\mathrm{d}\theta + \int_{0}^{2\pi}f(R,\theta;R)R\mathrm{d}\theta$$
这里$f^{'}$是关于$R$求导。$f$对$R$的依赖性可以看做$f$对$\Omega$的依赖性。例如,$f(\cdot;\Omega)\equiv f(\cdot;\Omega)$。使用笛卡尔坐标系(Cartesian coordinates)$\mathbf{x}$重新写上面公式有
\begin{equation}
\begin{aligned}
\frac{d}{dR}F=\int_{\Omega}f^{'}(\mathbf{x};\Omega)\mathrm{d}\mathbf{x} + \int_{\partial \Omega}f(\mathbf{x};\Omega)\mathrm{d}S(\mathbf{x})
\end{aligned}
\end{equation}
这里$\mathrm{d}\mathbf{x}$是体积测度,$\mathrm{d}S(\mathbf{x})$是表面积测度。

\subsubsection{一般扰动}
令$\boldsymbol{v}$是$\partial \Omega$的单位外法向量。我们把增加的$R$看做速度场(velocity field)$\mathbf{V}$,在法方向上$\mathbf{V}$驱动$\Omega$上的点,取在$\Omega$上$\mathbf{V}=\boldsymbol{v}$。因此,方程$(2)$变成
\begin{equation}
\begin{aligned}
\frac{d}{dR}F=\int_{\Omega}f^{'}(\mathbf{x};\Omega)\mathrm{d}\mathbf{x} + \int_{\partial \Omega}f(\mathbf{x};\Omega)\mathbf{V}(\mathbf{x})\cdot \boldsymbol{v}(\mathbf{x})\mathrm{d}S(\mathbf{x})
\end{aligned}
\end{equation}
我们把$\mathbf{V}$看成一个速度场,它会瞬间扰动(instantaneously perturbs domain)定义域$\Omega$。我们称$\mathbf{V}$是一个域扰动(domain perturbation)。现采用$f(\mathbf{x};\Omega)\equiv f(\Omega)$和$f^{'}(\mathbf{x};\Omega)\equiv f^{'}(\Omega;\mathbf{V})$,这里$f^{'}$是于域波动$\mathbf{V}$的$f$的形导数(shape derivative)。类似,我们定义$\delta F(\Omega;\mathbf{V})\equiv \frac{d}{dR}F$为$F$在方向$\mathbf{V}$上对于$\Omega$的形扰动(shape perturbation)。因此,我们有
\begin{equation}
\begin{aligned}
\delta F(\Omega;\mathbf{V})=\int_{\Omega}f^{'}(\Omega;\mathbf{V})+\int_{\partial \Omega}f(\Omega)(\mathbf{V} \cdot \boldsymbol{v}),
\end{aligned}
\end{equation}
这是第五章中公式$(5.47)$,
$$\delta \varepsilon (\Omega;\mathbf{V})=\int_{\Omega}\dot{f}(\Omega;\mathbf{V})+\int_{\Omega}f(\Omega)(\bigtriangledown \cdot \mathbf{V})=\int_{\Omega}f^{'}(\Omega;\mathbf{V})+\int_{\partial \Omega}f(\Omega)(\mathbf{V} \cdot \boldsymbol{v})$$因此,我们导出公式$(5.47)$,其中对于$\Omega$是一个被速度场$\mathbf{V}$扰动的圆盘,使$\Omega$沿着法线方向上均匀扩大。这本书的主要目的是对于一般区域$\Omega$和一般扰动$\mathbf{V}$导出方程$(4)$,和其他类似的公式。

形微分的框架提供了建立平均曲率流(mean curvature flow)和Willmore流方程的工具,它们是流体力学和生物学等许多应用中的几何流。见第六章和第七章例子。




\input{test.tex}



\nocite{*}
\cite{tam19912d}
%\bibliography{../ref}
\end{document}
